% generated by GAPDoc2LaTeX from XML source (Frank Luebeck)
\documentclass[a4paper,11pt]{report}

\usepackage{a4wide}
\sloppy
\pagestyle{myheadings}
\usepackage{amssymb}
\usepackage[utf8]{inputenc}
\usepackage{makeidx}
\makeindex
\usepackage{color}
\definecolor{FireBrick}{rgb}{0.5812,0.0074,0.0083}
\definecolor{RoyalBlue}{rgb}{0.0236,0.0894,0.6179}
\definecolor{RoyalGreen}{rgb}{0.0236,0.6179,0.0894}
\definecolor{RoyalRed}{rgb}{0.6179,0.0236,0.0894}
\definecolor{LightBlue}{rgb}{0.8544,0.9511,1.0000}
\definecolor{Black}{rgb}{0.0,0.0,0.0}

\definecolor{linkColor}{rgb}{0.0,0.0,0.554}
\definecolor{citeColor}{rgb}{0.0,0.0,0.554}
\definecolor{fileColor}{rgb}{0.0,0.0,0.554}
\definecolor{urlColor}{rgb}{0.0,0.0,0.554}
\definecolor{promptColor}{rgb}{0.0,0.0,0.589}
\definecolor{brkpromptColor}{rgb}{0.589,0.0,0.0}
\definecolor{gapinputColor}{rgb}{0.589,0.0,0.0}
\definecolor{gapoutputColor}{rgb}{0.0,0.0,0.0}

%%  for a long time these were red and blue by default,
%%  now black, but keep variables to overwrite
\definecolor{FuncColor}{rgb}{0.0,0.0,0.0}
%% strange name because of pdflatex bug:
\definecolor{Chapter }{rgb}{0.0,0.0,0.0}
\definecolor{DarkOlive}{rgb}{0.1047,0.2412,0.0064}


\usepackage{fancyvrb}

\usepackage{mathptmx,helvet}
\usepackage[T1]{fontenc}
\usepackage{textcomp}


\usepackage[
            pdftex=true,
            bookmarks=true,        
            a4paper=true,
            pdftitle={Written with GAPDoc},
            pdfcreator={LaTeX with hyperref package / GAPDoc},
            colorlinks=true,
            backref=page,
            breaklinks=true,
            linkcolor=linkColor,
            citecolor=citeColor,
            filecolor=fileColor,
            urlcolor=urlColor,
            pdfpagemode={UseNone}, 
           ]{hyperref}

\newcommand{\maintitlesize}{\fontsize{50}{55}\selectfont}

% write page numbers to a .pnr log file for online help
\newwrite\pagenrlog
\immediate\openout\pagenrlog =\jobname.pnr
\immediate\write\pagenrlog{PAGENRS := [}
\newcommand{\logpage}[1]{\protect\write\pagenrlog{#1, \thepage,}}
%% were never documented, give conflicts with some additional packages

\newcommand{\GAP}{\textsf{GAP}}

%% nicer description environments, allows long labels
\usepackage{enumitem}
\setdescription{style=nextline}

%% depth of toc
\setcounter{tocdepth}{1}





%% command for ColorPrompt style examples
\newcommand{\gapprompt}[1]{\color{promptColor}{\bfseries #1}}
\newcommand{\gapbrkprompt}[1]{\color{brkpromptColor}{\bfseries #1}}
\newcommand{\gapinput}[1]{\color{gapinputColor}{#1}}


\begin{document}

\logpage{[ 0, 0, 0 ]}
\begin{titlepage}
\mbox{}\vfill

\begin{center}{\maintitlesize \textbf{ Origami \mbox{}}}\\
\vfill

\hypersetup{pdftitle= Origami }
\markright{\scriptsize \mbox{}\hfill  Origami  \hfill\mbox{}}
{\Huge \textbf{ Computing Veechgroups of origamis \mbox{}}}\\
\vfill

{\Huge  1.0 \mbox{}}\\[1cm]
{ 26 February 2018 \mbox{}}\\[1cm]
\mbox{}\\[2cm]
{\Large \textbf{ Pascal Kattler\\
    \mbox{}}}\\
{\Large \textbf{ Sergio Siccha\\
    \mbox{}}}\\
{\Large \textbf{ Andrea Thevis\\
    \mbox{}}}\\
\hypersetup{pdfauthor= Pascal Kattler\\
    ;  Sergio Siccha\\
    ;  Andrea Thevis\\
    }
\end{center}\vfill

\mbox{}\\
{\mbox{}\\
\small \noindent \textbf{ Pascal Kattler\\
    }  Email: \href{mailto://kattler@math.uni-sb.de} {\texttt{kattler@math.uni-sb.de}}\\
  Homepage: \href{http://www.math.uni-sb.de/ag/weitze/} {\texttt{http://www.math.uni-sb.de/ag/weitze/}}\\
  Address: \begin{minipage}[t]{8cm}\noindent
 AG Weitze-Schmith{\"u}sen\\
 FR 6.1 Mathematik\\
 Universit{\"a}t des Saarlandes\\
 D-66041 Saarbr{\"u}cken\\
 \end{minipage}
}\\
{\mbox{}\\
\small \noindent \textbf{ Sergio Siccha\\
    }  Email: \href{mailto://siccha@mathb.rwth-aachen.de} {\texttt{siccha@mathb.rwth-aachen.de}}\\
  Homepage: \href{https://www.mathb.rwth-aachen.de/Mitarbeiter/siccha.php} {\texttt{https://www.mathb.rwth-aachen.de/Mitarbeiter/siccha.php}}\\
  Address: \begin{minipage}[t]{8cm}\noindent
 Lehrstuhl B f{\"u}r Mathematik RWTH - Aachen\\
 Pontdriesch 10-16\\
 52062 Aachen\\
 Germany\\
 \end{minipage}
}\\
{\mbox{}\\
\small \noindent \textbf{ Andrea Thevis\\
    }  Email: \href{mailto://thevis@math.uni-sb.de} {\texttt{thevis@math.uni-sb.de}}\\
  Homepage: \href{http://www.math.uni-sb.de/ag/weitze/} {\texttt{http://www.math.uni-sb.de/ag/weitze/}}\\
  Address: \begin{minipage}[t]{8cm}\noindent
 AG Weitze-Schmith{\"u}sen\\
 FR 6.1 Mathematik\\
 Universit{\"a}t des Saarlandes\\
 D-66041 Saarbr{\"u}cken\\
 \end{minipage}
}\\
\end{titlepage}

\newpage\setcounter{page}{2}
\newpage

\def\contentsname{Contents\logpage{[ 0, 0, 1 ]}}

\tableofcontents
\newpage

     
\chapter{\textcolor{Chapter }{The Origami object}}\label{Chapter_The_Origami_object}
\logpage{[ 1, 0, 0 ]}
\hyperdef{L}{X820B5C567EB3242E}{}
{
  
\section{\textcolor{Chapter }{The action on the Origami}}\label{Chapter_The_Origami_object_Section_The_action_on_the_Origami}
\logpage{[ 1, 1, 0 ]}
\hyperdef{L}{X783528BE8616720C}{}
{
  

\subsection{\textcolor{Chapter }{ActionOfSl}}
\logpage{[ 1, 1, 1 ]}\nobreak
\hyperdef{L}{X7C8464AA834B03C3}{}
{\noindent\textcolor{FuncColor}{$\triangleright$\enspace\texttt{ActionOfSl({\mdseries\slshape word, Origami})\index{ActionOfSl@\texttt{ActionOfSl}}
\label{ActionOfSl}
}\hfill{\scriptsize (function)}}\\
\textbf{\indent Returns:\ }
the Origami Object word.Origami 



 This lets act a word in the free group $Group(S, T)$ ,representing an element of $Sl_2(\mathbb{Z})$ on an Origami and returns $word.Origami$. }

 

\subsection{\textcolor{Chapter }{ActionOfF2ViaCanonical}}
\logpage{[ 1, 1, 2 ]}\nobreak
\hyperdef{L}{X7C16201B7EFC663B}{}
{\noindent\textcolor{FuncColor}{$\triangleright$\enspace\texttt{ActionOfF2ViaCanonical({\mdseries\slshape word, Origami})\index{ActionOfF2ViaCanonical@\texttt{ActionOfF2ViaCanonical}}
\label{ActionOfF2ViaCanonical}
}\hfill{\scriptsize (function)}}\\
\textbf{\indent Returns:\ }
the Origami Object word.Origami 



 This lets act a word in the free group $Group(S, T)$, representing an element of $Sl_2(\mathbb{Z})$, on an Origami and returns word.Origami. But in contrast to "ActionOfSl" the
result is stored in the canonical representation. }

 

\subsection{\textcolor{Chapter }{RightActionOfF2ViaCanonical}}
\logpage{[ 1, 1, 3 ]}\nobreak
\hyperdef{L}{X7C2684507C0FCD19}{}
{\noindent\textcolor{FuncColor}{$\triangleright$\enspace\texttt{RightActionOfF2ViaCanonical({\mdseries\slshape word, Origami})\index{RightActionOfF2ViaCanonical@\texttt{RightActionOfF2ViaCanonical}}
\label{RightActionOfF2ViaCanonical}
}\hfill{\scriptsize (function)}}\\
\textbf{\indent Returns:\ }
the Origami Object Origami.word 



 This lets act a word in the free group $Group(S, T)$ on an Origami from right and returns $Origami.word = word^-1.Origami$, where the left action is the common action of $Sl_2(\mathbb{Z})$ on 2 mannifolds. This action has the same Veechgroup and orbits as the left
action. In contrast to "ActionOfSl" the result is stored in the canonical
representation. }

 }

 
\section{\textcolor{Chapter }{The Origami object}}\label{Chapter_The_Origami_object_Section_The_Origami_object}
\logpage{[ 1, 2, 0 ]}
\hyperdef{L}{X820B5C567EB3242E}{}
{
  

\subsection{\textcolor{Chapter }{CanonicalOrigamiViaDelecroixAndStart}}
\logpage{[ 1, 2, 1 ]}\nobreak
\hyperdef{L}{X875D733E84F7C0BC}{}
{\noindent\textcolor{FuncColor}{$\triangleright$\enspace\texttt{CanonicalOrigamiViaDelecroixAndStart({\mdseries\slshape Origami, start})\index{CanonicalOrigamiViaDelecroixAndStart@\texttt{Canonical}\-\texttt{Origami}\-\texttt{Via}\-\texttt{Delecroix}\-\texttt{And}\-\texttt{Start}}
\label{CanonicalOrigamiViaDelecroixAndStart}
}\hfill{\scriptsize (function)}}\\
\textbf{\indent Returns:\ }
An Origami 



 This calculates a canonical representation of an origami depending on a given
number start (Between 1 and the degree of of the Origami). To determine a
canonical numbering the algorithm starts at the square with number start and
walks over the origami in a certain way and numbers the squares in the order,
they are visited . First it walks in horizontal direction one loop. Then it
walks one step up (in vertical direection) and then again a loop in horizontal
direction. This wil be repeated until the vertical loop is complete or all
squares have been visited. If there are unvisited squares, we continue with
the smallest number (in the new numbering), that has not been in a vertical
loop. An Origami is connected, so that number exists. Two origamis are equal
if they are described by the same permutations in their canonical
representation. }

 

\subsection{\textcolor{Chapter }{CanonicalOrigamiViaDelecroix}}
\logpage{[ 1, 2, 2 ]}\nobreak
\hyperdef{L}{X7E87D89D7F33C68B}{}
{\noindent\textcolor{FuncColor}{$\triangleright$\enspace\texttt{CanonicalOrigamiViaDelecroix({\mdseries\slshape Origami})\index{CanonicalOrigamiViaDelecroix@\texttt{CanonicalOrigamiViaDelecroix}}
\label{CanonicalOrigamiViaDelecroix}
}\hfill{\scriptsize (function)}}\\
\textbf{\indent Returns:\ }
An Origami 



 This calculates a canonical representation of an origami. It calculates the
representation from CanonicalOrigamiViaDelecroixAndStart with several start
squares, independent of the given representation. Then it takes the minimum
with respect to some order. Two origamis are equal if they are described by
the same permutations in their canonical representation. }

 

\subsection{\textcolor{Chapter }{CanonicalOrigami}}
\logpage{[ 1, 2, 3 ]}\nobreak
\hyperdef{L}{X85B271997F1D6583}{}
{\noindent\textcolor{FuncColor}{$\triangleright$\enspace\texttt{CanonicalOrigami({\mdseries\slshape Origami})\index{CanonicalOrigami@\texttt{CanonicalOrigami}}
\label{CanonicalOrigami}
}\hfill{\scriptsize (function)}}\\
\textbf{\indent Returns:\ }
An Origami 



 This calculates a canonical representation of an origami, represented as
record rec(d := * , x := *, y := *). Two origamis are equal if they are
described by the same permutations in their canonical representation. }

 

\subsection{\textcolor{Chapter }{VerticalPerm (for IsOrigami)}}
\logpage{[ 1, 2, 4 ]}\nobreak
\hyperdef{L}{X7F00ED687F561955}{}
{\noindent\textcolor{FuncColor}{$\triangleright$\enspace\texttt{VerticalPerm({\mdseries\slshape Origami})\index{VerticalPerm@\texttt{VerticalPerm}!for IsOrigami}
\label{VerticalPerm:for IsOrigami}
}\hfill{\scriptsize (attribute)}}\\
\textbf{\indent Returns:\ }
a permutation 



 This returns the horizontal permutation $\sigma_x$ of the Origami. }

 

\subsection{\textcolor{Chapter }{HorizontalPerm (for IsOrigami)}}
\logpage{[ 1, 2, 5 ]}\nobreak
\hyperdef{L}{X845B15517C65A113}{}
{\noindent\textcolor{FuncColor}{$\triangleright$\enspace\texttt{HorizontalPerm({\mdseries\slshape Origami})\index{HorizontalPerm@\texttt{HorizontalPerm}!for IsOrigami}
\label{HorizontalPerm:for IsOrigami}
}\hfill{\scriptsize (attribute)}}\\
\textbf{\indent Returns:\ }
a permutation 



 This returns the vertical permutation $\sigma_y$ of the Origami. }

 

\subsection{\textcolor{Chapter }{DegreeOrigami (for IsOrigami)}}
\logpage{[ 1, 2, 6 ]}\nobreak
\hyperdef{L}{X87EABE8F83D207F7}{}
{\noindent\textcolor{FuncColor}{$\triangleright$\enspace\texttt{DegreeOrigami({\mdseries\slshape Origami})\index{DegreeOrigami@\texttt{DegreeOrigami}!for IsOrigami}
\label{DegreeOrigami:for IsOrigami}
}\hfill{\scriptsize (attribute)}}\\
\textbf{\indent Returns:\ }
an integer 



 This returns the degree of an Origami. }

 

\subsection{\textcolor{Chapter }{Stratum (for IsOrigami)}}
\logpage{[ 1, 2, 7 ]}\nobreak
\hyperdef{L}{X7D7005357B19F69D}{}
{\noindent\textcolor{FuncColor}{$\triangleright$\enspace\texttt{Stratum({\mdseries\slshape Origami})\index{Stratum@\texttt{Stratum}!for IsOrigami}
\label{Stratum:for IsOrigami}
}\hfill{\scriptsize (attribute)}}\\
\textbf{\indent Returns:\ }
a list of integers 



 This calculates the stratum of an Origami. That is a list of the orders of the
singularities. }

 

\subsection{\textcolor{Chapter }{VeechGroup (for IsOrigami)}}
\logpage{[ 1, 2, 8 ]}\nobreak
\hyperdef{L}{X83B46D177B01C205}{}
{\noindent\textcolor{FuncColor}{$\triangleright$\enspace\texttt{VeechGroup({\mdseries\slshape Origami})\index{VeechGroup@\texttt{VeechGroup}!for IsOrigami}
\label{VeechGroup:for IsOrigami}
}\hfill{\scriptsize (attribute)}}\\
\textbf{\indent Returns:\ }
a ModularSubgroup object 



 This calculates the Veechgroup of an Origami. This is a subgroup of $Sl_2(\mathbb{Z})$ of finite degree. The group is stored as ModularSubgroup from the
ModularSubgroup package. The Veechgroup is represented as the coset
permutations $\sigma_S$ and $\sigma_T$ with respect to the generators $S$ and $T$. This means if $i$ is the integer associated to the rigth coset $G$ (Cosets( O ) [ i ] VeechGroup = H) then we have for the coset H, associated to $\sigma_S(i)$, that $SG = H$. Dito for $\sigma_T$. }

 

\subsection{\textcolor{Chapter }{Cosets (for IsOrigami)}}
\logpage{[ 1, 2, 9 ]}\nobreak
\hyperdef{L}{X79C9587E81B84172}{}
{\noindent\textcolor{FuncColor}{$\triangleright$\enspace\texttt{Cosets({\mdseries\slshape Origami})\index{Cosets@\texttt{Cosets}!for IsOrigami}
\label{Cosets:for IsOrigami}
}\hfill{\scriptsize (attribute)}}\\
\textbf{\indent Returns:\ }
a list of words in the Free group, generated by $S$ and $T$. 



 This Calculates the right cosets of the Veechgroup of an Origami. }

 

\subsection{\textcolor{Chapter }{Equals}}
\logpage{[ 1, 2, 10 ]}\nobreak
\hyperdef{L}{X7B9F20F283BE1AFB}{}
{\noindent\textcolor{FuncColor}{$\triangleright$\enspace\texttt{Equals({\mdseries\slshape Origami1, Origami2})\index{Equals@\texttt{Equals}}
\label{Equals}
}\hfill{\scriptsize (function)}}\\
\textbf{\indent Returns:\ }
true or false 



 This tests wether Origami1 is equal to Origami2 with same numbering of
squares. That is, the defining permutations are the same. }

 

\subsection{\textcolor{Chapter }{EquivalentOrigami}}
\logpage{[ 1, 2, 11 ]}\nobreak
\hyperdef{L}{X7E5CC3BB87F11E78}{}
{\noindent\textcolor{FuncColor}{$\triangleright$\enspace\texttt{EquivalentOrigami({\mdseries\slshape Origami1, Origami2})\index{EquivalentOrigami@\texttt{EquivalentOrigami}}
\label{EquivalentOrigami}
}\hfill{\scriptsize (function)}}\\
\textbf{\indent Returns:\ }
true or false 



 This tests wether Origami1 is equal up to Origami2 up to numbering of the
squares. }

 

\subsection{\textcolor{Chapter }{ExampleOrigami}}
\logpage{[ 1, 2, 12 ]}\nobreak
\hyperdef{L}{X82128FEE7CFC18E8}{}
{\noindent\textcolor{FuncColor}{$\triangleright$\enspace\texttt{ExampleOrigami({\mdseries\slshape d})\index{ExampleOrigami@\texttt{ExampleOrigami}}
\label{ExampleOrigami}
}\hfill{\scriptsize (function)}}\\
\textbf{\indent Returns:\ }
a random origami 



 This creates a random origami of degree d. }

 

\subsection{\textcolor{Chapter }{CalcVeechGroup}}
\logpage{[ 1, 2, 13 ]}\nobreak
\hyperdef{L}{X7D6E8BA479D1222B}{}
{\noindent\textcolor{FuncColor}{$\triangleright$\enspace\texttt{CalcVeechGroup({\mdseries\slshape Origami})\index{CalcVeechGroup@\texttt{CalcVeechGroup}}
\label{CalcVeechGroup}
}\hfill{\scriptsize (function)}}\\
\textbf{\indent Returns:\ }
A list with tree entrys 



 This function is used to calculate some attributes. It calculates the
Veechgroup of a given origami and . the veechgroup is stored as ModularGroup
Object from the ModularGroup package. The cosets of the veechgroup is stored
in a list of words in the generators S and T of the matrix group
Sl{\textunderscore}2(Z). }

 

\subsection{\textcolor{Chapter }{CalcVeechGroupViaEquivalentTest}}
\logpage{[ 1, 2, 14 ]}\nobreak
\hyperdef{L}{X82311344837D478A}{}
{\noindent\textcolor{FuncColor}{$\triangleright$\enspace\texttt{CalcVeechGroupViaEquivalentTest({\mdseries\slshape Origami})\index{CalcVeechGroupViaEquivalentTest@\texttt{CalcVeechGroupViaEquivalentTest}}
\label{CalcVeechGroupViaEquivalentTest}
}\hfill{\scriptsize (function)}}\\
\textbf{\indent Returns:\ }
A list with tree entrys 



 This function is used to calculate some attributes. It calculates the
Veechgroup of a given origami and . the veechgroup is stored as ModularGroup
Object from the ModularGroup package. The cosets of the veechgroup is stored
in a list of words in the generators S and T of the matrix group
Sl{\textunderscore}2(Z). In Contrast to CalcVeechGroup, this uses equivalent
tests instead of canonical Origamis. }

 

\subsection{\textcolor{Chapter }{CalcVeechGroupWithHashTables}}
\logpage{[ 1, 2, 15 ]}\nobreak
\hyperdef{L}{X7D70AD7387B53621}{}
{\noindent\textcolor{FuncColor}{$\triangleright$\enspace\texttt{CalcVeechGroupWithHashTables({\mdseries\slshape Origami})\index{CalcVeechGroupWithHashTables@\texttt{CalcVeechGroupWithHashTables}}
\label{CalcVeechGroupWithHashTables}
}\hfill{\scriptsize (function)}}\\
\textbf{\indent Returns:\ }
A list with tree entrys 



 This function is used to calculate some attributes. It calculates the
Veechgroup of a given origami and . the veechgroup is stored as ModularGroup
Object from the ModularGroup package. The cosets of the veechgroup is stored
in a list of words in the generators S and T of the matrix group
Sl{\textunderscore}2(Z). In contrast to CalcVeechGroup, this uses hash tables
to store Origamis. }

 

\subsection{\textcolor{Chapter }{CalcStratum}}
\logpage{[ 1, 2, 16 ]}\nobreak
\hyperdef{L}{X7D1805ED812B3514}{}
{\noindent\textcolor{FuncColor}{$\triangleright$\enspace\texttt{CalcStratum({\mdseries\slshape Origami})\index{CalcStratum@\texttt{CalcStratum}}
\label{CalcStratum}
}\hfill{\scriptsize (function)}}\\
\textbf{\indent Returns:\ }
nothing 



 Calculates the stratum of an object and sets its attribute. The stratum is
stored as list of integers. }

 

\subsection{\textcolor{Chapter }{ToRec}}
\logpage{[ 1, 2, 17 ]}\nobreak
\hyperdef{L}{X7880174884C40B23}{}
{\noindent\textcolor{FuncColor}{$\triangleright$\enspace\texttt{ToRec({\mdseries\slshape Origami})\index{ToRec@\texttt{ToRec}}
\label{ToRec}
}\hfill{\scriptsize (function)}}\\
\textbf{\indent Returns:\ }
record of the form rec(d := * , x := *, y := *) Describtion This calculates a
record representation for an origami object. 



 }

 }

 }

   
\chapter{\textcolor{Chapter }{Introduction}}\label{Chapter_Introduction}
\logpage{[ 2, 0, 0 ]}
\hyperdef{L}{X7DFB63A97E67C0A1}{}
{
  This package provides calculations with Origamis. An Origami can be obtained
in the following way from two permutations $\sigma_a, \sigma_b \in S_d$. We take $d$ Squares $Q_1, \dots, Q_d$ and clue the lower side of $Q_i$ to the upper side of $Q_{\sigma_y(i)}$ and the right side of $Q_i$ to the left side of $Q_{\sigma_x(i)}$. So in this Package we identify an Origami with a pair of permutations, witch
acts transitive on $\{1 \dots d\}$ up to simultan conjugation. We store an Origami as Origami object. We are
mainly interested in the Veechgroup of an Origami. It can be shown that the
Veechgroup of an Origami is a subgroup of $Sl_2(\mathbb{Z})$ of finite index. So we fix two generators 
$$
S =
\left( {\begin{array}{cc}
0 & -1 \\
1 & 0 \\
\end{array} } \right)
$$
 and 
$$
T =
\left( {\begin{array}{cc}
1 & 1 \\
0 & 1 \\
\end{array} } \right).
$$
 
\section{\textcolor{Chapter }{The Free Group}}\label{Chapter_Introduction_Section_The_Free_Group}
\logpage{[ 2, 1, 0 ]}
\hyperdef{L}{X7EFD570179465DD5}{}
{
  In this package we fix the Free Group $F$ generated by $S$ and $T$. }

 }

 \def\indexname{Index\logpage{[ "Ind", 0, 0 ]}
\hyperdef{L}{X83A0356F839C696F}{}
}

\cleardoublepage
\phantomsection
\addcontentsline{toc}{chapter}{Index}


\printindex

\immediate\write\pagenrlog{["Ind", 0, 0], \arabic{page},}
\newpage
\immediate\write\pagenrlog{["End"], \arabic{page}];}
\immediate\closeout\pagenrlog
\end{document}
