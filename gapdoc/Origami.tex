% generated by GAPDoc2LaTeX from XML source (Frank Luebeck)
\documentclass[a4paper,11pt]{report}

\usepackage{a4wide}
\sloppy
\pagestyle{myheadings}
\usepackage{amssymb}
\usepackage[latin1]{inputenc}
\usepackage{makeidx}
\makeindex
\usepackage{color}
\definecolor{FireBrick}{rgb}{0.5812,0.0074,0.0083}
\definecolor{RoyalBlue}{rgb}{0.0236,0.0894,0.6179}
\definecolor{RoyalGreen}{rgb}{0.0236,0.6179,0.0894}
\definecolor{RoyalRed}{rgb}{0.6179,0.0236,0.0894}
\definecolor{LightBlue}{rgb}{0.8544,0.9511,1.0000}
\definecolor{Black}{rgb}{0.0,0.0,0.0}

\definecolor{linkColor}{rgb}{0.0,0.0,0.554}
\definecolor{citeColor}{rgb}{0.0,0.0,0.554}
\definecolor{fileColor}{rgb}{0.0,0.0,0.554}
\definecolor{urlColor}{rgb}{0.0,0.0,0.554}
\definecolor{promptColor}{rgb}{0.0,0.0,0.589}
\definecolor{brkpromptColor}{rgb}{0.589,0.0,0.0}
\definecolor{gapinputColor}{rgb}{0.589,0.0,0.0}
\definecolor{gapoutputColor}{rgb}{0.0,0.0,0.0}

%%  for a long time these were red and blue by default,
%%  now black, but keep variables to overwrite
\definecolor{FuncColor}{rgb}{0.0,0.0,0.0}
%% strange name because of pdflatex bug:
\definecolor{Chapter }{rgb}{0.0,0.0,0.0}
\definecolor{DarkOlive}{rgb}{0.1047,0.2412,0.0064}


\usepackage{fancyvrb}

\usepackage{mathptmx,helvet}
\usepackage[T1]{fontenc}
\usepackage{textcomp}


\usepackage[
            pdftex=true,
            bookmarks=true,        
            a4paper=true,
            pdftitle={Written with GAPDoc},
            pdfcreator={LaTeX with hyperref package / GAPDoc},
            colorlinks=true,
            backref=page,
            breaklinks=true,
            linkcolor=linkColor,
            citecolor=citeColor,
            filecolor=fileColor,
            urlcolor=urlColor,
            pdfpagemode={UseNone}, 
           ]{hyperref}

\newcommand{\maintitlesize}{\fontsize{50}{55}\selectfont}

% write page numbers to a .pnr log file for online help
\newwrite\pagenrlog
\immediate\openout\pagenrlog =\jobname.pnr
\immediate\write\pagenrlog{PAGENRS := [}
\newcommand{\logpage}[1]{\protect\write\pagenrlog{#1, \thepage,}}
%% were never documented, give conflicts with some additional packages

\newcommand{\GAP}{\textsf{GAP}}

%% nicer description environments, allows long labels
\usepackage{enumitem}
\setdescription{style=nextline}

%% depth of toc
\setcounter{tocdepth}{1}





%% command for ColorPrompt style examples
\newcommand{\gapprompt}[1]{\color{promptColor}{\bfseries #1}}
\newcommand{\gapbrkprompt}[1]{\color{brkpromptColor}{\bfseries #1}}
\newcommand{\gapinput}[1]{\color{gapinputColor}{#1}}


\begin{document}

\logpage{[ 0, 0, 0 ]}
\begin{titlepage}
\mbox{}\vfill

\begin{center}{\maintitlesize \textbf{The \textsf{Origami} Package\mbox{}}}\\
\vfill

\hypersetup{pdftitle=The \textsf{Origami} Package}
\markright{\scriptsize \mbox{}\hfill The \textsf{Origami} Package \hfill\mbox{}}
{\Huge \textbf{Computing Veechgroups of Origamis\mbox{}}}\\
\vfill

{\Huge Version 1.0.0\mbox{}}\\[1cm]
{13.11.2018\mbox{}}\\[1cm]
\mbox{}\\[2cm]
{\Large \textbf{Pascal Kattler    \mbox{}}}\\
{\Large \textbf{Andrea Thevis    \mbox{}}}\\
\hypersetup{pdfauthor=Pascal Kattler    ; Andrea Thevis    }
\end{center}\vfill

\mbox{}\\
{\mbox{}\\
\small \noindent \textbf{Pascal Kattler    }  Email: \href{mailto://kattler@math.uni-sb.de} {\texttt{kattler@math.uni-sb.de}}\\
  Homepage: \href{http://www.math.uni-sb.de/ag/weitze/} {\texttt{http://www.math.uni-sb.de/ag/weitze/}}\\
  Address: \begin{minipage}[t]{8cm}\noindent
 AG Weitze-Schmith{\"u}sen\\
 FR 6.1 Mathematik\\
 Universit{\"a}t des Saarlandes\\
 D-66041 Saarbr{\"u}cken\\
 \end{minipage}
}\\
{\mbox{}\\
\small \noindent \textbf{Andrea Thevis    }  Email: \href{mailto://thevis@math.uni-sb.de} {\texttt{thevis@math.uni-sb.de}}\\
  Homepage: \href{http://www.math.uni-sb.de/ag/weitze/} {\texttt{http://www.math.uni-sb.de/ag/weitze/}}\\
  Address: \begin{minipage}[t]{8cm}\noindent
 AG Weitze-Schmith{\"u}sen\\
 FR 6.1 Mathematik\\
 Universit{\"a}t des Saarlandes\\
 D-66041 Saarbr{\"u}cken\\
 \end{minipage}
}\\
\end{titlepage}

\newpage\setcounter{page}{2}
{\small 
\section*{Copyright}
\logpage{[ 0, 0, 1 ]}
{\copyright} 2018 by Pascal Kattler \mbox{}}\\[1cm]
{\small 
\section*{Acknowledgements}
\logpage{[ 0, 0, 2 ]}
 Supported by Project I.8 of SFB-TRR 195 'Symbolic Tools in Mathematics and
their Application' of the German Research Foundation (DFG). \mbox{}}\\[1cm]
\newpage

\def\contentsname{Contents\logpage{[ 0, 0, 3 ]}}

\tableofcontents
\newpage

 
\chapter{\textcolor{Chapter }{The Veechgroup of Origamis }}\logpage{[ 1, 0, 0 ]}
\hyperdef{L}{X8298DEB886A4B953}{}
{
  
\section{\textcolor{Chapter }{Introduction}}\label{sec:theory}
\logpage{[ 1, 1, 0 ]}
\hyperdef{L}{X7DFB63A97E67C0A1}{}
{
  This package provides calculations with origamis. An origami (also known as
square-tiled surface) is a finite covering of a torus which is ramified at
most over one point. It can be described in the following way from two
permutations $\sigma_x, \sigma_y \in S_d$. We take $d$ squares $Q_1, \dots, Q_d$ and glue the lower side of $Q_i$ to the upper side of $Q_{\sigma_y(i)}$ and the right side of $Q_i$ to the left side of $Q_{\sigma_x(i)}$. We require origamis to be connected and thus the group generated by $\sigma_x$ and $\sigma_y$ acts transitively on $\{1,\dots,d\}$. In this package we identify an origami with a pair of permutations, which
acts transitively on $\{1, \dots, d\}$ up to simultaneous conjugation. We introduce a new type of origamis, namely
origami objects, which are created by this two permutations and its degree.
The degree of an origami is the number of squares. Origamis are stored as such
objects. We are mainly interested in the Veechgroup of an origami. It can be
shown that the Veechgroup of an origami is a subgroup of $SL_2(\mathbb{Z})$ of finite index. So we fix two generators of $SL_2(\mathbb{Z})$ 
\[ S = \left( {\begin{array}{cc} 0 & -1 \\ 1 & 0 \\ \end{array} } \right) \]
 and 
\[ T = \left( {\begin{array}{cc} 1 & 1 \\ 0 & 1 \\ \end{array} } \right). \]
 }

 
\section{\textcolor{Chapter }{ The Free Group }}\logpage{[ 1, 2, 0 ]}
\hyperdef{L}{X7EFD570179465DD5}{}
{
  In this package we fix the free group $F$ generated by $\tilde{S}$ and $\tilde{T}$. We consider the canonical epimorphism $\pi: F\to SL_2(\mathbb{Z})$ with $\pi(\tilde{S})=S$ and $\pi(\tilde{T})=T$. }

 
\section{\textcolor{Chapter }{The Origami Object}}\logpage{[ 1, 3, 0 ]}
\hyperdef{L}{X820B5C567EB3242E}{}
{
  In this section we describe the main functions of this package. 

\subsection{\textcolor{Chapter }{Origami}}
\logpage{[ 1, 3, 1 ]}\nobreak
\hyperdef{L}{X80F0ED3B78404E71}{}
{\noindent\textcolor{FuncColor}{$\triangleright$\enspace\texttt{Origami({\mdseries\slshape permX, permy, d})\index{Origami@\texttt{Origami}}
\label{Origami}
}\hfill{\scriptsize (function)}}\\


 This function generates a new origami object with $ \sigma_x =$ \mbox{\texttt{\mdseries\slshape  permX }}, $ \sigma_y =$ \mbox{\texttt{\mdseries\slshape  permY }} and degree \mbox{\texttt{\mdseries\slshape d}}. 
\begin{Verbatim}[commandchars=!@|,fontsize=\small,frame=single,label=Example]
  !gapprompt@gap>| !gapinput@Origami((1,2), (2,3), 3);
|
  Origami((1,2), (2,3), 3)
          
\end{Verbatim}
 \noindent\textcolor{FuncColor}{$\triangleright$\enspace\texttt{OrigamiWithoutTest({\mdseries\slshape permX, permy, d})\index{OrigamiWithoutTest@\texttt{OrigamiWithoutTest}}
\label{OrigamiWithoutTest}
}\hfill{\scriptsize (function)}}\\


 This function does the same as \texttt{Origami}, but in contrast it does not test, weather the origami describes a connected
surface. \noindent\textcolor{FuncColor}{$\triangleright$\enspace\texttt{ActionOfSpecialLinearGroup({\mdseries\slshape matrix, origami})\index{ActionOfSpecialLinearGroup@\texttt{ActionOfSpecialLinearGroup}}
\label{ActionOfSpecialLinearGroup}
}\hfill{\scriptsize (operation)}}\\


 The group $SL_2(\mathbb{Z}) $ acts on the set of origamis of fixed degree. The follwoing methods for this
operation are installed. \noindent\textcolor{FuncColor}{$\triangleright$\enspace\texttt{ActionOfSpecialLinearGroup({\mdseries\slshape word, origami})\index{ActionOfSpecialLinearGroup@\texttt{ActionOfSpecialLinearGroup}}
\label{ActionOfSpecialLinearGroup}
}\hfill{\scriptsize (method)}}\\


 Given a word \mbox{\texttt{\mdseries\slshape word}} in the free group $Group(\tilde{S}, \tilde{T})$ this function computes $\pi(\mbox{\texttt{\mdseries\slshape word}}) \in SL_2(\mathbb{Z})$ and returns $\pi(\mbox{\texttt{\mdseries\slshape word}}).\mbox{\texttt{\mdseries\slshape origami}}$. The word is given as a string, as shown in the following example. 
\begin{Verbatim}[commandchars=!@|,fontsize=\small,frame=single,label=Example]
  !gapprompt@gap>| !gapinput@ActionOfSpecialLinearGroup("ST",Origami((1,3,5), (1,3)(2,4,5), 5));
|
  Origami((1,3)(2,5,4), (2,4,5,3), 5)
\end{Verbatim}
 \noindent\textcolor{FuncColor}{$\triangleright$\enspace\texttt{ActionOfSpecialLinearGroup({\mdseries\slshape matrix, origami})\index{ActionOfSpecialLinearGroup@\texttt{ActionOfSpecialLinearGroup}}
\label{ActionOfSpecialLinearGroup}
}\hfill{\scriptsize (method)}}\\


 Given matrix in $SL_2(\mathbb{Z}) $ this function returns \mbox{\texttt{\mdseries\slshape matrix}}.\mbox{\texttt{\mdseries\slshape origami}}. The word is given as a string, as shown in the following example. 
\begin{Verbatim}[commandchars=!@|,fontsize=\small,frame=single,label=Example]
  !gapprompt@gap>| !gapinput@ActionOfSpecialLinearGroup([ [ 0, -1 ], [ 1, 1 ] ], Origami((1,3,5), 
|
  >(1,3)(2,4,5), 5)); 
  Origami((1,3)(2,5,4), (2,4,5,3), 5)
\end{Verbatim}
 \noindent\textcolor{FuncColor}{$\triangleright$\enspace\texttt{ActionOfF2ViaCanonical({\mdseries\slshape origami, word})\index{ActionOfF2ViaCanonical@\texttt{ActionOfF2ViaCanonical}}
\label{ActionOfF2ViaCanonical}
}\hfill{\scriptsize (function)}}\\


 Given a word \mbox{\texttt{\mdseries\slshape word}} in the free group $Group(\tilde{S}, \tilde{T})$ this function computes $\pi(\mbox{\texttt{\mdseries\slshape word}}) \in SL_2(\mathbb{Z})$ and returns $\pi(\mbox{\texttt{\mdseries\slshape word}}).\mbox{\texttt{\mdseries\slshape origami}}$. But in contrast to \texttt{ActionOfSpecialLinearGroup} the result is stored in the canonical representation. ATTENTION: the order of
arguments is switched compared to the order of the arguments in the function \texttt{ActionOfSpecialLinearGroup}. 
\begin{Verbatim}[commandchars=!@|,fontsize=\small,frame=single,label=Example]
  !gapprompt@gap>| !gapinput@ActionOfF2ViaCanonical(Origami((1,2), (1,3), 3), "S");
|
  Origami((1,2), (2,3), 3)
    
\end{Verbatim}
 \noindent\textcolor{FuncColor}{$\triangleright$\enspace\texttt{RightActionOfF2ViaCanonical({\mdseries\slshape origami, word})\index{RightActionOfF2ViaCanonical@\texttt{RightActionOfF2ViaCanonical}}
\label{RightActionOfF2ViaCanonical}
}\hfill{\scriptsize (function)}}\\


 This function computes the right action of the projection of a word \mbox{\texttt{\mdseries\slshape word}} in the free group $Group(\tilde{S}, \tilde{T})$ on an origami \mbox{\texttt{\mdseries\slshape origami}}. It returns $\mbox{\texttt{\mdseries\slshape origami}}.\pi(\mbox{\texttt{\mdseries\slshape word}}) = \pi(\mbox{\texttt{\mdseries\slshape word}})^-1.\mbox{\texttt{\mdseries\slshape origami}}$, where the left action is the common action of $SL_2(\mathbb{Z})$ on origamis of a given degree. This action has the same orbits as the left
action. For the Veechgroup computation both actions can be used and give the
same result. In contrast to \texttt{ActionOfSpecialLinearGroup} the result is stored in the canonical representation. ATTENTION: the order of
arguments is switched compared to the order of the arguments in the function \texttt{ActionOfSpecialLinearGroup}. 
\begin{Verbatim}[commandchars=!@|,fontsize=\small,frame=single,label=Example]
  !gapprompt@gap>| !gapinput@RightActionOfF2ViaCanonical(Origami((2,3), (1,3,2), 3),"T");
|
  Origami((1,2), (2,3), 3)
\end{Verbatim}
 \noindent\textcolor{FuncColor}{$\triangleright$\enspace\texttt{CanonicalOrigamiViaDelecroixAndStart({\mdseries\slshape origami, start})\index{CanonicalOrigamiViaDelecroixAndStart@\texttt{Canonical}\-\texttt{Origami}\-\texttt{Via}\-\texttt{Delecroix}\-\texttt{And}\-\texttt{Start}}
\label{CanonicalOrigamiViaDelecroixAndStart}
}\hfill{\scriptsize (function)}}\\


 This function calculates a canonical representation of an origami depending on
a given number start (Between 1 and the degree of of the origami). To
determine a canonical numbering the algorithm starts at the square with number
start. This sqare is labeled as 1 in the new numbering. The algorithm walks
along the origami in the following way and numbers the squares in the order,
they are visited. First it walks in horizontal direction until it reaches the
square with number start again. Then it walks one step up (in vertical
direction) and then again a loop in horizontal direction. This will be
repeated until the vertical loop is complete or all squares have been visited.
If there are unvisited squares, we continue with the smallest number (in the
new numbering), that has not been in a vertical loop. An origami is connected,
so that number exists. This function is used to determine a canonical origami
independent of the starting point. The idea for the algorithm derives from the
algorithm to{\textunderscore}standard{\textunderscore}form implemented in the
SageMath package surface{\textunderscore}dynamics. (REFERENZ HINZUF{\"U}GEN!) 
\begin{Verbatim}[commandchars=!@|,fontsize=\small,frame=single,label=Example]
  !gapprompt@gap>| !gapinput@CanonicalOrigamiAndStart(Origami((1,10,7,6,8,9,2)(4,5),
|
  !gapprompt@>| !gapinput@(1,8,5,4)(2,10,6)(3,9,7), 10), 1);
|
  Origami((1,2,3,4,5,6,7)(8,9), (1,5,8,9)(2,4,7)(3,10,6), 10)
\end{Verbatim}
 \noindent\textcolor{FuncColor}{$\triangleright$\enspace\texttt{CanonicalOrigamiViaDelecroix({\mdseries\slshape origami})\index{CanonicalOrigamiViaDelecroix@\texttt{CanonicalOrigamiViaDelecroix}}
\label{CanonicalOrigamiViaDelecroix}
}\hfill{\scriptsize (function)}}\\


 This function computes a canonical representation of an origami. It calculates
the representation from CanonicalOrigamiViaDelecroixAndStart with all squares
as start squares, independent of the given representation. Then it takes the
minimum with respect to the order which \textsf{GAP} automatically uses to compare paires of permutations. Two origamis are equal
(up to renumbering of the squares) if they are described by the same
permutations in their canonical representation. 
\begin{Verbatim}[commandchars=!@|,fontsize=\small,frame=single,label=Example]
  !gapprompt@gap>| !gapinput@CanonicalOrigami(Origami((1,10,7,6,8,9,2)(4,5), (1,8,5,4)(2,10,6)(3,9,7), 10));
|
  Origami((2,3,4,5,6,7,8)(9,10), (1,2,6)(3,5,7)(4,8,9,10), 10)
\end{Verbatim}
 \noindent\textcolor{FuncColor}{$\triangleright$\enspace\texttt{OrigamiFamily\index{OrigamiFamily@\texttt{OrigamiFamily}}
\label{OrigamiFamily}
}\hfill{\scriptsize (family)}}\\


 Since origamis do not fit in any existing family in \textsf{GAP}, we introduce a new family for origami objects called OrigamiFamily. \noindent\textcolor{FuncColor}{$\triangleright$\enspace\texttt{HorizontalPerm({\mdseries\slshape origami})\index{HorizontalPerm@\texttt{HorizontalPerm}}
\label{HorizontalPerm}
}\hfill{\scriptsize (attribute)}}\\


 This function returns the horizontal permutation $\sigma_y$ of an origami \mbox{\texttt{\mdseries\slshape origami}}. 
\begin{Verbatim}[commandchars=!@|,fontsize=\small,frame=single,label=Example]
  !gapprompt@gap>| !gapinput@HorizontalPerm(Origami((1,3,5), (1,3)(2,4,5), 5));
|
  (1,3,5)
\end{Verbatim}
 \noindent\textcolor{FuncColor}{$\triangleright$\enspace\texttt{VerticalPerm({\mdseries\slshape origami})\index{VerticalPerm@\texttt{VerticalPerm}}
\label{VerticalPerm}
}\hfill{\scriptsize (attribute)}}\\


 This function returns the vertical permutation $\sigma_x$ of an origami \mbox{\texttt{\mdseries\slshape origami}}. 
\begin{Verbatim}[commandchars=!@|,fontsize=\small,frame=single,label=Example]
  !gapprompt@gap>| !gapinput@VerticalPerm( Origami((1,3,5), (1,3)(2,4,5), 5));
|
  (1,3)(2,4,5)
\end{Verbatim}
 \noindent\textcolor{FuncColor}{$\triangleright$\enspace\texttt{DegreeOrigami({\mdseries\slshape origami})\index{DegreeOrigami@\texttt{DegreeOrigami}}
\label{DegreeOrigami}
}\hfill{\scriptsize (attribute)}}\\


 This function returns the degree of an \mbox{\texttt{\mdseries\slshape origami}}. 
\begin{Verbatim}[commandchars=!@|,fontsize=\small,frame=single,label=Example]
  !gapprompt@gap>| !gapinput@DegreeOrigami(Origami((1,3,5), (1,3)(2,4,5), 5));
|
  5
\end{Verbatim}
 \noindent\textcolor{FuncColor}{$\triangleright$\enspace\texttt{Stratum({\mdseries\slshape origami})\index{Stratum@\texttt{Stratum}}
\label{Stratum}
}\hfill{\scriptsize (attribute)}}\\


 This function calculates the stratum of an origami \mbox{\texttt{\mdseries\slshape origami}}. The stratum of an origami is a list of the nonzero degrees of the
singularities. For a singularity with cone angle $2\pi k$ the degree of the singularity is $k-1$. 
\begin{Verbatim}[commandchars=!@|,fontsize=\small,frame=single,label=Example]
  !gapprompt@gap>| !gapinput@Stratum(Origami((1,6,4,7,5,3)(2,8), (1,4,5,3,8,2,6), 8));
|
  [ 1, 5 ]
\end{Verbatim}
 \noindent\textcolor{FuncColor}{$\triangleright$\enspace\texttt{Genus({\mdseries\slshape origami})\index{Genus@\texttt{Genus}}
\label{Genus}
}\hfill{\scriptsize (attribute)}}\\


 This function calculates the genus of the origami surface. 
\begin{Verbatim}[commandchars=!@|,fontsize=\small,frame=single,label=Example]
  !gapprompt@gap>| !gapinput@Genus( Origami((1,2,3,4),(1,2)(3,4), 4) );
|
  2
\end{Verbatim}
 \noindent\textcolor{FuncColor}{$\triangleright$\enspace\texttt{VeechGroup({\mdseries\slshape origami})\index{VeechGroup@\texttt{VeechGroup}}
\label{VeechGroup}
}\hfill{\scriptsize (attribute)}}\\


 This function calculates the Veechgroup of an origami \mbox{\texttt{\mdseries\slshape origami}}. The Veechgroup $G$ of \mbox{\texttt{\mdseries\slshape origami}} is a subgroup $SL_2(\mathbb{Z})$ of finite index. The group is stored as a ModularSubgroup from the \textsf{ModularSubgroup} package. It is represented by two permutations $\sigma_S$ and $\sigma_T$ describing how the generators $S$ and $T$ of $SL_2(\mathbb{Z})$ act on the cosets of $G$ in $SL_2(\mathbb{Z})$. E.g, if $SH_i = H_j$ and $H_i,H_j$ are the cosets associated to the integers $i,j$, respectively, then $\sigma_S(i)=j$. The algorithm was introduced by Gabriela Weitze-Schmith{\"u}sen in \cite{weitze_schmithuesen_phd}. You get the coset permutations using the \textsf{ModularSubgroup} package as in the following example. 
\begin{Verbatim}[commandchars=!@|,fontsize=\small,frame=single,label=Example]
  !gapprompt@gap>| !gapinput@SAction(VeechGroup(Origami((1,2,5)(3,4,6), (1,2)(5,6), 6)));
|
  (1,3)(2,5)(4,7)(6,8)(9,10)
  !gapprompt@gap>| !gapinput@TAction(VeechGroup(Origami((1,2,5)(3,4,6), (1,2)(5,6), 6)));
|
  (1,2,4)(3,6)(5,8,7,9,10)
          
\end{Verbatim}
 \noindent\textcolor{FuncColor}{$\triangleright$\enspace\texttt{Cosets({\mdseries\slshape origami})\index{Cosets@\texttt{Cosets}}
\label{Cosets}
}\hfill{\scriptsize (attribute)}}\\


 This function calculates the right cosets of the Veechgroup of an origami \mbox{\texttt{\mdseries\slshape origami}} as a list of words in $S$ and $T$. 
\begin{Verbatim}[commandchars=!@|,fontsize=\small,frame=single,label=Example]
  !gapprompt@gap>| !gapinput@Cosets(Origami((1,2,5)(3,4,6), (1,2)(5,6), 6));
|
  [ <identity ...>, S, T, T^-1, S*T, S*T^-1, T*S, T^-1*S, S*T*S, S*T^-1*S ]
\end{Verbatim}
 \noindent\textcolor{FuncColor}{$\triangleright$\enspace\texttt{EquivalentOrigami({\mdseries\slshape origami1, origami2})\index{EquivalentOrigami@\texttt{EquivalentOrigami}}
\label{EquivalentOrigami}
}\hfill{\scriptsize (function)}}\\


 This function tests whether \mbox{\texttt{\mdseries\slshape origami1}} is equal to \mbox{\texttt{\mdseries\slshape origami2}} up to renumbering of the squares. 
\begin{Verbatim}[commandchars=!@|,fontsize=\small,frame=single,label=Example]
  !gapprompt@gap>| !gapinput@EquivalentOrigami(Origami((1,4)(2,6,3), (1,5)(2,3,6,4), 6), Origami((1,4,3)
|
  >(2,5), (1,5,3,4)(2,6), 6));
  true
  
  !gapprompt@gap>| !gapinput@EquivalentOrigami(Origami((1,4)(2,6,3), (1,5)(2,3,6,4), 6), Origami((1,2,5)
|
  >(3,4,6), (1,2)(5,6), 6));
  false
\end{Verbatim}
 }

 }

 
\section{\textcolor{Chapter }{List of Origamis}}\logpage{[ 1, 4, 0 ]}
\hyperdef{L}{X86D3DB9D82100CED}{}
{
  The following functions generarte lists of Origamis. 

\subsection{\textcolor{Chapter }{OrigamiList}}
\logpage{[ 1, 4, 1 ]}\nobreak
\hyperdef{L}{X7CED322D7B64C6C3}{}
{\noindent\textcolor{FuncColor}{$\triangleright$\enspace\texttt{OrigamiList({\mdseries\slshape d})\index{OrigamiList@\texttt{OrigamiList}}
\label{OrigamiList}
}\hfill{\scriptsize (function)}}\\


 This Function calculates a list of all origamis, with a given degree \mbox{\texttt{\mdseries\slshape d}}. 
\begin{Verbatim}[commandchars=!@|,fontsize=\small,frame=single,label=Example]
  !gapprompt@gap>| !gapinput@OrigamiList(2);
|
  [ Origami((), (1,2), 2), Origami((1,2), (), 2), Origami((1,2), (1,2), 2) ]
\end{Verbatim}
 \noindent\textcolor{FuncColor}{$\triangleright$\enspace\texttt{OrigamiListWithStratum({\mdseries\slshape d, stratum})\index{OrigamiListWithStratum@\texttt{OrigamiListWithStratum}}
\label{OrigamiListWithStratum}
}\hfill{\scriptsize (function)}}\\


 This function calculates a list of all origamis with a given degree \mbox{\texttt{\mdseries\slshape d}} and stratum \mbox{\texttt{\mdseries\slshape stratum}}. 
\begin{Verbatim}[commandchars=!@|,fontsize=\small,frame=single,label=Example]
  !gapprompt@gap>| !gapinput@OrigamiListWithStratum(5, [1,1]);
|
  [ Origami((1,2), (1,3)(2,4,5), 5), Origami((1,2), (1,3,2,4,5), 5), 
    Origami((1,2,3), (1,2)(3,4,5), 5), Origami((1,2,3), (2,3,4,5), 5), 
    Origami((1,2,3), (2,4,5,3), 5), Origami((1,2)(3,4,5), (2,3), 5), 
    Origami((1,2)(3,4,5), (1,2,3), 5), Origami((1,2)(3,4,5), (2,3,4,5), 5), 
    Origami((1,2)(3,4,5), (2,3,5,4), 5), Origami((1,2)(3,4,5), (1,2,3,4,5), 5), 
    Origami((1,2)(3,4,5), (1,2,3,5,4), 5), Origami((1,2,3,4), (3,4,5), 5), 
    Origami((1,2,3,4), (3,5,4), 5), Origami((1,2,3,4), (1,2,3)(4,5), 5), 
    Origami((1,2,3,4), (1,3,2)(4,5), 5), Origami((1,2,3,4), (2,3,4,5), 5), 
    Origami((1,2,3,4), (2,5,4,3), 5), Origami((1,2,3,4), (1,2,4,5,3), 5), 
    Origami((1,2,3,4), (1,3,5,4,2), 5), Origami((1,2,3,4,5), (3,5), 5), 
    Origami((1,2,3,4,5), (1,2)(3,4,5), 5), Origami((1,2,3,4,5), (1,2)(3,5,4), 5)
      , Origami((1,2,3,4,5), (2,4,3,5), 5), Origami((1,2,3,4,5), (2,5,3,4), 5) ]
\end{Verbatim}
 }

 }

 
\section{\textcolor{Chapter }{ Using SageMath functions}}\logpage{[ 1, 5, 0 ]}
\hyperdef{L}{X868D83F186ECD853}{}
{
  The SageMath Package surface{\textunderscore}dynamics from Vincent Delecroix
provides methods for origamis. To use the functions of this chapter, Sage must
be installed on your operation system. 

\subsection{\textcolor{Chapter }{VeechgroupBySage}}
\logpage{[ 1, 5, 1 ]}\nobreak
\hyperdef{L}{X7983495981F365DF}{}
{\noindent\textcolor{FuncColor}{$\triangleright$\enspace\texttt{VeechgroupBySage({\mdseries\slshape origami})\index{VeechgroupBySage@\texttt{VeechgroupBySage}}
\label{VeechgroupBySage}
}\hfill{\scriptsize (function)}}\\


 This function executes the SageMath method Veechgroup to \mbox{\texttt{\mdseries\slshape origami}} and returns its result as \textsf{GAP} object. 
\begin{Verbatim}[commandchars=!@|,fontsize=\small,frame=single,label=Example]
  
\end{Verbatim}
 \noindent\textcolor{FuncColor}{$\triangleright$\enspace\texttt{NormalFormBySage({\mdseries\slshape origami})\index{NormalFormBySage@\texttt{NormalFormBySage}}
\label{NormalFormBySage}
}\hfill{\scriptsize (function)}}\\


 This function executes the SageMath method NormalForm to \mbox{\texttt{\mdseries\slshape origami}} and returns its result as \textsf{GAP} object. 
\begin{Verbatim}[commandchars=!@|,fontsize=\small,frame=single,label=Example]
  
\end{Verbatim}
 \noindent\textcolor{FuncColor}{$\triangleright$\enspace\texttt{IsHyperellipticBySage({\mdseries\slshape origami})\index{IsHyperellipticBySage@\texttt{IsHyperellipticBySage}}
\label{IsHyperellipticBySage}
}\hfill{\scriptsize (function)}}\\


 This function executes the SageMath method IsHyperelliptic to \mbox{\texttt{\mdseries\slshape origami}} and returns its result as \textsf{GAP} object. 
\begin{Verbatim}[commandchars=!@|,fontsize=\small,frame=single,label=Example]
  
\end{Verbatim}
 \noindent\textcolor{FuncColor}{$\triangleright$\enspace\texttt{IsPrimitiveBySage({\mdseries\slshape origami})\index{IsPrimitiveBySage@\texttt{IsPrimitiveBySage}}
\label{IsPrimitiveBySage}
}\hfill{\scriptsize (function)}}\\


 This function executes the SageMath method IsPrimitive to \mbox{\texttt{\mdseries\slshape origami}} and returns its result as \textsf{GAP} object. 
\begin{Verbatim}[commandchars=!@|,fontsize=\small,frame=single,label=Example]
  
\end{Verbatim}
 \noindent\textcolor{FuncColor}{$\triangleright$\enspace\texttt{ReduceBySage({\mdseries\slshape origami})\index{ReduceBySage@\texttt{ReduceBySage}}
\label{ReduceBySage}
}\hfill{\scriptsize (function)}}\\


 This function executes the SageMath method Reduce to \mbox{\texttt{\mdseries\slshape origami}} and returns its result as \textsf{GAP} object. 
\begin{Verbatim}[commandchars=!@|,fontsize=\small,frame=single,label=Example]
  
\end{Verbatim}
 \noindent\textcolor{FuncColor}{$\triangleright$\enspace\texttt{AbsolutePeriodGeneratorsBySage({\mdseries\slshape origami})\index{AbsolutePeriodGeneratorsBySage@\texttt{AbsolutePeriodGeneratorsBySage}}
\label{AbsolutePeriodGeneratorsBySage}
}\hfill{\scriptsize (function)}}\\


 This function executes the SageMath method AbsolutePeriodGenerators to \mbox{\texttt{\mdseries\slshape origami}} and returns its result as \textsf{GAP} object. 
\begin{Verbatim}[commandchars=!@|,fontsize=\small,frame=single,label=Example]
  
\end{Verbatim}
 \noindent\textcolor{FuncColor}{$\triangleright$\enspace\texttt{LatticeOfAbsolutePeriodsBySage({\mdseries\slshape origami})\index{LatticeOfAbsolutePeriodsBySage@\texttt{LatticeOfAbsolutePeriodsBySage}}
\label{LatticeOfAbsolutePeriodsBySage}
}\hfill{\scriptsize (function)}}\\


 This function executes the SageMath method LatticeOfAbsolutePeriods to \mbox{\texttt{\mdseries\slshape origami}} and returns its result as \textsf{GAP} object. 
\begin{Verbatim}[commandchars=!@|,fontsize=\small,frame=single,label=Example]
  
\end{Verbatim}
 \noindent\textcolor{FuncColor}{$\triangleright$\enspace\texttt{LatticeOfQuotientsBySage({\mdseries\slshape origami})\index{LatticeOfQuotientsBySage@\texttt{LatticeOfQuotientsBySage}}
\label{LatticeOfQuotientsBySage}
}\hfill{\scriptsize (function)}}\\


 This function executes the SageMath method LatticeOfQuotients to \mbox{\texttt{\mdseries\slshape origami}} and returns its result as \textsf{GAP} object. 
\begin{Verbatim}[commandchars=!@|,fontsize=\small,frame=single,label=Example]
  
\end{Verbatim}
 \noindent\textcolor{FuncColor}{$\triangleright$\enspace\texttt{OptimalDegreeBySage({\mdseries\slshape origami})\index{OptimalDegreeBySage@\texttt{OptimalDegreeBySage}}
\label{OptimalDegreeBySage}
}\hfill{\scriptsize (function)}}\\


 This function executes the SageMath method OptimalDegree to \mbox{\texttt{\mdseries\slshape origami}} and returns its result as \textsf{GAP} object. 
\begin{Verbatim}[commandchars=!@|,fontsize=\small,frame=single,label=Example]
  
\end{Verbatim}
 \noindent\textcolor{FuncColor}{$\triangleright$\enspace\texttt{PeriodGeneratorsBySage({\mdseries\slshape origami})\index{PeriodGeneratorsBySage@\texttt{PeriodGeneratorsBySage}}
\label{PeriodGeneratorsBySage}
}\hfill{\scriptsize (function)}}\\


 This function executes the SageMath method PeriodGenerators to \mbox{\texttt{\mdseries\slshape origami}} and returns its result as \textsf{GAP} object. 
\begin{Verbatim}[commandchars=!@|,fontsize=\small,frame=single,label=Example]
  
\end{Verbatim}
 \noindent\textcolor{FuncColor}{$\triangleright$\enspace\texttt{WidthsAndHeightsBySage({\mdseries\slshape origami})\index{WidthsAndHeightsBySage@\texttt{WidthsAndHeightsBySage}}
\label{WidthsAndHeightsBySage}
}\hfill{\scriptsize (function)}}\\


 This function executes the SageMath method WidthsAndHeights to \mbox{\texttt{\mdseries\slshape origami}} and returns its result as \textsf{GAP} object. 
\begin{Verbatim}[commandchars=!@|,fontsize=\small,frame=single,label=Example]
  
\end{Verbatim}
 \noindent\textcolor{FuncColor}{$\triangleright$\enspace\texttt{SumOfLyapunovExponentsBySage({\mdseries\slshape origami})\index{SumOfLyapunovExponentsBySage@\texttt{SumOfLyapunovExponentsBySage}}
\label{SumOfLyapunovExponentsBySage}
}\hfill{\scriptsize (function)}}\\


 This function executes the SageMath method SumOfLyapunovExponents to \mbox{\texttt{\mdseries\slshape origami}} and returns its result as \textsf{GAP} object. 
\begin{Verbatim}[commandchars=!@|,fontsize=\small,frame=single,label=Example]
  
\end{Verbatim}
 \noindent\textcolor{FuncColor}{$\triangleright$\enspace\texttt{LyapunovExponentsApproxBySage({\mdseries\slshape origami})\index{LyapunovExponentsApproxBySage@\texttt{LyapunovExponentsApproxBySage}}
\label{LyapunovExponentsApproxBySage}
}\hfill{\scriptsize (function)}}\\


 This function executes the SageMath method LyapunovExponentsApprox to \mbox{\texttt{\mdseries\slshape origami}} and returns its result as \textsf{GAP} object. 
\begin{Verbatim}[commandchars=!@|,fontsize=\small,frame=single,label=Example]
  
\end{Verbatim}
 \noindent\textcolor{FuncColor}{$\triangleright$\enspace\texttt{IntermediateCoversBySage({\mdseries\slshape origami})\index{IntermediateCoversBySage@\texttt{IntermediateCoversBySage}}
\label{IntermediateCoversBySage}
}\hfill{\scriptsize (function)}}\\


 This function executes the SageMath method NormalFormBySage to \mbox{\texttt{\mdseries\slshape origami}} and returns its result as \textsf{GAP} object. 
\begin{Verbatim}[commandchars=!@|,fontsize=\small,frame=single,label=Example]
  
\end{Verbatim}
 }

 }

 }

 \def\bibname{References\logpage{[ "Bib", 0, 0 ]}
\hyperdef{L}{X7A6F98FD85F02BFE}{}
}

\bibliographystyle{alpha}
\bibliography{Origami}

\addcontentsline{toc}{chapter}{References}

\def\indexname{Index\logpage{[ "Ind", 0, 0 ]}
\hyperdef{L}{X83A0356F839C696F}{}
}

\cleardoublepage
\phantomsection
\addcontentsline{toc}{chapter}{Index}


\printindex

\newpage
\immediate\write\pagenrlog{["End"], \arabic{page}];}
\immediate\closeout\pagenrlog
\end{document}
