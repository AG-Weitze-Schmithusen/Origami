% generated by GAPDoc2LaTeX from XML source (Frank Luebeck)
\documentclass[a4paper,11pt]{report}

\usepackage{a4wide}
\sloppy
\pagestyle{myheadings}
\usepackage{amssymb}
\usepackage[latin1]{inputenc}
\usepackage{makeidx}
\makeindex
\usepackage{color}
\definecolor{FireBrick}{rgb}{0.5812,0.0074,0.0083}
\definecolor{RoyalBlue}{rgb}{0.0236,0.0894,0.6179}
\definecolor{RoyalGreen}{rgb}{0.0236,0.6179,0.0894}
\definecolor{RoyalRed}{rgb}{0.6179,0.0236,0.0894}
\definecolor{LightBlue}{rgb}{0.8544,0.9511,1.0000}
\definecolor{Black}{rgb}{0.0,0.0,0.0}

\definecolor{linkColor}{rgb}{0.0,0.0,0.554}
\definecolor{citeColor}{rgb}{0.0,0.0,0.554}
\definecolor{fileColor}{rgb}{0.0,0.0,0.554}
\definecolor{urlColor}{rgb}{0.0,0.0,0.554}
\definecolor{promptColor}{rgb}{0.0,0.0,0.589}
\definecolor{brkpromptColor}{rgb}{0.589,0.0,0.0}
\definecolor{gapinputColor}{rgb}{0.589,0.0,0.0}
\definecolor{gapoutputColor}{rgb}{0.0,0.0,0.0}

%%  for a long time these were red and blue by default,
%%  now black, but keep variables to overwrite
\definecolor{FuncColor}{rgb}{0.0,0.0,0.0}
%% strange name because of pdflatex bug:
\definecolor{Chapter }{rgb}{0.0,0.0,0.0}
\definecolor{DarkOlive}{rgb}{0.1047,0.2412,0.0064}


\usepackage{fancyvrb}

\usepackage{mathptmx,helvet}
\usepackage[T1]{fontenc}
\usepackage{textcomp}


\usepackage[
            pdftex=true,
            bookmarks=true,        
            a4paper=true,
            pdftitle={Written with GAPDoc},
            pdfcreator={LaTeX with hyperref package / GAPDoc},
            colorlinks=true,
            backref=page,
            breaklinks=true,
            linkcolor=linkColor,
            citecolor=citeColor,
            filecolor=fileColor,
            urlcolor=urlColor,
            pdfpagemode={UseNone}, 
           ]{hyperref}

\newcommand{\maintitlesize}{\fontsize{50}{55}\selectfont}

% write page numbers to a .pnr log file for online help
\newwrite\pagenrlog
\immediate\openout\pagenrlog =\jobname.pnr
\immediate\write\pagenrlog{PAGENRS := [}
\newcommand{\logpage}[1]{\protect\write\pagenrlog{#1, \thepage,}}
%% were never documented, give conflicts with some additional packages

\newcommand{\GAP}{\textsf{GAP}}

%% nicer description environments, allows long labels
\usepackage{enumitem}
\setdescription{style=nextline}

%% depth of toc
\setcounter{tocdepth}{1}





%% command for ColorPrompt style examples
\newcommand{\gapprompt}[1]{\color{promptColor}{\bfseries #1}}
\newcommand{\gapbrkprompt}[1]{\color{brkpromptColor}{\bfseries #1}}
\newcommand{\gapinput}[1]{\color{gapinputColor}{#1}}


\begin{document}

\logpage{[ 0, 0, 0 ]}
\begin{titlepage}
\mbox{}\vfill

\begin{center}{\maintitlesize \textbf{The \textsf{Origami} Package\mbox{}}}\\
\vfill

\hypersetup{pdftitle=The \textsf{Origami} Package}
\markright{\scriptsize \mbox{}\hfill The \textsf{Origami} Package \hfill\mbox{}}
{\Huge \textbf{Computiong Veechgroup of origamis\mbox{}}}\\
\vfill

{\Huge Version UNKNOWNEntity(VERSION)\mbox{}}\\[1cm]
{UNKNOWNEntity(RELEASEDATE)\mbox{}}\\[1cm]
\mbox{}\\[2cm]
{\Large \textbf{Pascal Kattler    \mbox{}}}\\
{\Large \textbf{Andrea Thevis    \mbox{}}}\\
\hypersetup{pdfauthor=Pascal Kattler    ; Andrea Thevis    }
\end{center}\vfill

\mbox{}\\
{\mbox{}\\
\small \noindent \textbf{Pascal Kattler    }  Email: \href{mailto://kattler@math.uni-sb.de} {\texttt{kattler@math.uni-sb.de}}\\
  Homepage: \href{http://www.math.uni-sb.de/ag/weitze/} {\texttt{http://www.math.uni-sb.de/ag/weitze/}}\\
  Address: \begin{minipage}[t]{8cm}\noindent
 AG Weitze-Schmith{\"u}sen\\
 FR 6.1 Mathematik\\
 Universit{\"a}t des Saarlandes\\
 D-66041 Saarbr{\"u}cken\\
 \end{minipage}
}\\
{\mbox{}\\
\small \noindent \textbf{Andrea Thevis    }  Email: \href{mailto://thevis@math.uni-sb.de} {\texttt{thevis@math.uni-sb.de}}\\
  Homepage: \href{http://www.math.uni-sb.de/ag/weitze/} {\texttt{http://www.math.uni-sb.de/ag/weitze/}}\\
  Address: \begin{minipage}[t]{8cm}\noindent
 AG Weitze-Schmith{\"u}sen\\
 FR 6.1 Mathematik\\
 Universit{\"a}t des Saarlandes\\
 D-66041 Saarbr{\"u}cken\\
 \end{minipage}
}\\
\end{titlepage}

\newpage\setcounter{page}{2}
{\small 
\section*{Copyright}
\logpage{[ 0, 0, 1 ]}
{\copyright} UNKNOWNEntity(RELEASEYEAR) by Pascal Kattler \mbox{}}\\[1cm]
{\small 
\section*{Acknowledgements}
\logpage{[ 0, 0, 2 ]}
 Supported by Project I.8 of SFB-TRR 195 'Symbolic Tools in Mathematics and
their Application' of the German Research Foundation (DFG). \mbox{}}\\[1cm]
\newpage

\def\contentsname{Contents\logpage{[ 0, 0, 3 ]}}

\tableofcontents
\newpage

 
\chapter{\textcolor{Chapter }{The Veechgroup of origamis }}\logpage{[ 1, 0, 0 ]}
\hyperdef{L}{X8298DEB886A4B953}{}
{
  
\section{\textcolor{Chapter }{Introduction}}\label{sec:theory}
\logpage{[ 1, 1, 0 ]}
\hyperdef{L}{X7DFB63A97E67C0A1}{}
{
  This package provides calculations with origamis. An origami can be obtained
in the following way from two permutations $\sigma_a, \sigma_b \in S_d$. We take $d$ squares $Q_1, \dots, Q_d$ and glue the lower side of $Q_i$ to the upper side of $Q_{\sigma_y(i)}$ and the right side of $Q_i$ to the left side of $Q_{\sigma_x(i)}$. So in this package we identify an origami with a pair of permutations, which
acts transitively on $\{1 \dots d\}$ up to simultaneous conjugation. We introduce a new type of origamis, namely
origami objects, which are created by this two permutations and its degree.
The degree of an origami is the number of squares. Origamis are stored as such
objects. We are mainly interested in the Veechgroup of an origami. It can be
shown that the Veechgroup of an origami is a subgroup of $SL_2(\mathbb{Z})$ of finite index. So we fix two generators of $SL_2(\mathbb{Z})$ 
\[ S = \left( {\begin{array}{cc} 0 & -1 \\ 1 & 0 \\ \end{array} } \right) \]
 and 
\[ T = \left( {\begin{array}{cc} 1 & 1 \\ 0 & 1 \\ \end{array} } \right). \]
 }

 
\section{\textcolor{Chapter }{ The Free Group }}\logpage{[ 1, 2, 0 ]}
\hyperdef{L}{X7EFD570179465DD5}{}
{
  In this package we fix the free group $F$ generated by $\tilde{S}$ and $\tilde{T}$. }

 
\section{\textcolor{Chapter }{The Origmai Object}}\logpage{[ 1, 3, 0 ]}
\hyperdef{L}{X7DCBDD6F8173A517}{}
{
  In this section we describe the main function of this package. 

\subsection{\textcolor{Chapter }{ActionOfSl}}
\logpage{[ 1, 3, 1 ]}\nobreak
\hyperdef{L}{X7C8464AA834B03C3}{}
{\noindent\textcolor{FuncColor}{$\triangleright$\enspace\texttt{ActionOfSl({\mdseries\slshape word, origami})\index{ActionOfSl@\texttt{ActionOfSl}}
\label{ActionOfSl}
}\hfill{\scriptsize (function)}}\\


 The group $SL_2(\mathbb{Z}) $ acts on the set of origamis with a fixednumber of squares. This function lets
act a word in the free group $Group(\tilde{S}, \tilde{T})$ ,representing an element of $Sl_2(\mathbb{Z})$ on an origami and returns $word.origami$. The word is given as string, as you can see in the following example. 
\begin{Verbatim}[commandchars=!@|,fontsize=\small,frame=single,label=Example]
  !gapprompt@gap>| !gapinput@ActionOfSl("ST",Origami((1,3,5), (1,3)(2,4,5), 5));|
  Origami((1,3)(2,5,4), (2,4,5,3), 5)
\end{Verbatim}
 \noindent\textcolor{FuncColor}{$\triangleright$\enspace\texttt{ActionOfF2ViaCanonical({\mdseries\slshape origami, word})\index{ActionOfF2ViaCanonical@\texttt{ActionOfF2ViaCanonical}}
\label{ActionOfF2ViaCanonical}
}\hfill{\scriptsize (function)}}\\


 This lets act a word in the free group $Group(S, T)$, representing an element of $Sl_2(\mathbb{Z})$, on an Origami and returns $word.Origami$. But in contrast to \texttt{ActionOfSl} the result is stored in the canonical representation. ATTENTION: the order of
arguments is here reserved. 
\begin{Verbatim}[commandchars=!@|,fontsize=\small,frame=single,label=Example]
  !gapprompt@gap>| !gapinput@ActionOfF2ViaCanonical(Origami((1,2), (1,3), 3), "S");|
  Origami((1,2), (2,3), 3)
    
\end{Verbatim}
 \noindent\textcolor{FuncColor}{$\triangleright$\enspace\texttt{RightActionOfF2ViaCanonical({\mdseries\slshape origami, word})\index{RightActionOfF2ViaCanonical@\texttt{RightActionOfF2ViaCanonical}}
\label{RightActionOfF2ViaCanonical}
}\hfill{\scriptsize (function)}}\\


 This lets act a word in the free group $Group(S, T)$ on an Origami from right and returns $Origami.word = word^-1.Origami$, where the left action is the common action of $Sl_2(\mathbb{Z})$ on 2 mannifolds. This action has the same Veechgroup and orbits as the left
action. In contrast to \texttt{ActionOfSl} the result is stored in the canonical representation. ATTENTION: the order of
arguments is here reserved. 
\begin{Verbatim}[commandchars=!@|,fontsize=\small,frame=single,label=Example]
  !gapprompt@gap>| !gapinput@RightActionOfF2ViaCanonical(Origami((2,3), (1,3,2), 3),"T");|
  Origami((1,2), (2,3), 3)
   
\end{Verbatim}
 \noindent\textcolor{FuncColor}{$\triangleright$\enspace\texttt{CanonicalOrigamiViaDelecroixAndStart({\mdseries\slshape origami, start})\index{CanonicalOrigamiViaDelecroixAndStart@\texttt{Canonical}\-\texttt{Origami}\-\texttt{Via}\-\texttt{Delecroix}\-\texttt{And}\-\texttt{Start}}
\label{CanonicalOrigamiViaDelecroixAndStart}
}\hfill{\scriptsize (function)}}\\


 This function calculates a canonical representation of an origami depending on
a given number start (Between 1 and the degree of of the origami). To
determine a canonical numbering the algorithm starts at the square with number
start. This sqare is labeled 1 in the new numbering. The algorithm walks along
the origami in the following way and numbers the squares in the order, they
are visited. First it walks in horizontal direction until it reaches the
square with number start again. Then it walks one step up (in vertical
direection) and then again a loop in horizontal direction. This wil be
repeated until the vertical loop is complete or all squares have been visited.
If there are unvisited squares, we continue with the smallest number (in the
new numbering), that has not been in a vertical loop. An origami is connected,
so that number exists. This function is used to determine a canonical origami
independent of the start. 
\begin{Verbatim}[commandchars=!@|,fontsize=\small,frame=single,label=Example]
  !gapprompt@gap>| !gapinput@CanonicalOrigamiAndStart(Origami((1,10,7,6,8,9,2)(4,5),|
  !gapprompt@>| !gapinput@(1,8,5,4)(2,10,6)(3,9,7), 10), 1);|
  Origami((1,2,3,4,5,6,7)(8,9), (1,5,8,9)(2,4,7)(3,10,6), 10)
\end{Verbatim}
 \noindent\textcolor{FuncColor}{$\triangleright$\enspace\texttt{CanonicalOrigamiViaDelecroix({\mdseries\slshape origami})\index{CanonicalOrigamiViaDelecroix@\texttt{CanonicalOrigamiViaDelecroix}}
\label{CanonicalOrigamiViaDelecroix}
}\hfill{\scriptsize (function)}}\\


 This calculates a canonical representation of an origami. It calculates the
representation from CanonicalOrigamiViaDelecroixAndStart with all squares as
start squares, independent of the given representation. Then it takes the
minimum with respect to the order which \textsf{GAP} automatically uses to compare paires of permutations. Two origamis are equal
if they are described by the same permutations in their canonical
representation. 
\begin{Verbatim}[commandchars=!@|,fontsize=\small,frame=single,label=Example]
  !gapprompt@gap>| !gapinput@CanonicalOrigami(Origami((1,10,7,6,8,9,2)(4,5), (1,8,5,4)(2,10,6)(3,9,7), 10));|
  Origami((2,3,4,5,6,7,8)(9,10), (1,2,6)(3,5,7)(4,8,9,10), 10)
\end{Verbatim}
 \noindent\textcolor{FuncColor}{$\triangleright$\enspace\texttt{OrigamiFamily\index{OrigamiFamily@\texttt{OrigamiFamily}}
\label{OrigamiFamily}
}\hfill{\scriptsize (family)}}\\


 The only sense of this familiy is, that origami does not fit in any other
family. \noindent\textcolor{FuncColor}{$\triangleright$\enspace\texttt{HorizontalPerm({\mdseries\slshape origami})\index{HorizontalPerm@\texttt{HorizontalPerm}}
\label{HorizontalPerm}
}\hfill{\scriptsize (attribute)}}\\


 This function returns the vertical permutation $\sigma_y$ of the origami. 
\begin{Verbatim}[commandchars=!@|,fontsize=\small,frame=single,label=Example]
          gap> HorizontalPerm(Origami((1,3,5), (1,3)(2,4,5), 5));
          (1,3,5)
          
\end{Verbatim}
 \noindent\textcolor{FuncColor}{$\triangleright$\enspace\texttt{VerticalPerm({\mdseries\slshape origami})\index{VerticalPerm@\texttt{VerticalPerm}}
\label{VerticalPerm}
}\hfill{\scriptsize (attribute)}}\\


 This function returns the horizontal permutation $\sigma_x$ of the origami. 
\begin{Verbatim}[commandchars=!@|,fontsize=\small,frame=single,label=Example]
          gap> VerticalPerm( Origami((1,3,5), (1,3)(2,4,5), 5));
          (1,3)(2,4,5)
          
\end{Verbatim}
 \noindent\textcolor{FuncColor}{$\triangleright$\enspace\texttt{DegreeOrigami({\mdseries\slshape origami})\index{DegreeOrigami@\texttt{DegreeOrigami}}
\label{DegreeOrigami}
}\hfill{\scriptsize (attribute)}}\\


 This function returns the degree of an origami. 
\begin{Verbatim}[commandchars=!@|,fontsize=\small,frame=single,label=Example]
          gap> DegreeOrigami(Origami((1,3,5), (1,3)(2,4,5), 5));
          5
          
\end{Verbatim}
 \noindent\textcolor{FuncColor}{$\triangleright$\enspace\texttt{Stratum({\mdseries\slshape Origami})\index{Stratum@\texttt{Stratum}}
\label{Stratum}
}\hfill{\scriptsize (attribute)}}\\


 This function calculates the stratum of an Origami. That is a list of the
degrees of the singularities 
\begin{Verbatim}[commandchars=!@|,fontsize=\small,frame=single,label=Example]
          gap> Stratum(Origami((1,6,4,7,5,3)(2,8), (1,4,5,3,8,2,6), 8));
          [ 1, 5 ]
          
\end{Verbatim}
 \noindent\textcolor{FuncColor}{$\triangleright$\enspace\texttt{Genus({\mdseries\slshape origami})\index{Genus@\texttt{Genus}}
\label{Genus}
}\hfill{\scriptsize (attribute)}}\\


 This function calculates the genus of the origami surface. 
\begin{Verbatim}[commandchars=!@|,fontsize=\small,frame=single,label=Example]
          gap> Genus( Origami((1,2,3,4),(1,2)(3,4), 4) );
          2
          
\end{Verbatim}
 \noindent\textcolor{FuncColor}{$\triangleright$\enspace\texttt{VeechGroup({\mdseries\slshape origami})\index{VeechGroup@\texttt{VeechGroup}}
\label{VeechGroup}
}\hfill{\scriptsize (attribute)}}\\


 This function calculates the Veechgroup of an origami. This is a subgroup $SL_2(\mathbb{Z})$ of finite index. The group is stored as a ModularSubgroup from the \textsf{ModularSubgroup} package. The Veechgroup is represented as the coset permutations $\sigma_S$ and $\sigma_T$ with respect to the generators $S$ and $T$. This means if $i$ is the integer associated to the right coset $G$ (Cosets( O ) [ i ] VeechGroup = H) then we have for the coset $H$, associated to $\sigma_S(i)$, that $SG = H$. Analogously for $\sigma_T$. You get the coset Permutations from the ModularSubgroup as in the following
example. 
\begin{Verbatim}[commandchars=!@|,fontsize=\small,frame=single,label=Example]
  !gapprompt@gap>| !gapinput@SAction(VeechGroup(Origami((1,2,5)(3,4,6), (1,2)(5,6), 6)));|
  (1,3)(2,5)(4,7)(6,8)(9,10)
  !gapprompt@gap>| !gapinput@TAction(VeechGroup(Origami((1,2,5)(3,4,6), (1,2)(5,6), 6)));|
  (1,2,4)(3,6)(5,8,7,9,10)
          
\end{Verbatim}
 \noindent\textcolor{FuncColor}{$\triangleright$\enspace\texttt{Cosets({\mdseries\slshape origami})\index{Cosets@\texttt{Cosets}}
\label{Cosets}
}\hfill{\scriptsize (attribute)}}\\


 This function calculates the right cosets of the Veechgroup of an origami as a
list of words in $S$ and $T$. 
\begin{Verbatim}[commandchars=!@|,fontsize=\small,frame=single,label=Example]
          gap> Cosets(Origami((1,2,5)(3,4,6), (1,2)(5,6), 6));
          [ < identity ...>, T, S, T^2, T*S, S*T, T^2*S, T*S*T, T^2*S*T, T^2*S*T^2 ]
          
\end{Verbatim}
 \noindent\textcolor{FuncColor}{$\triangleright$\enspace\texttt{EquivalentOrigami({\mdseries\slshape origami1, origami2})\index{EquivalentOrigami@\texttt{EquivalentOrigami}}
\label{EquivalentOrigami}
}\hfill{\scriptsize (function)}}\\


 This function tests wether origami1 is equal to origami2 up to renumbering of
the squares. 
\begin{Verbatim}[commandchars=!@|,fontsize=\small,frame=single,label=Example]
          gap> EquivalentOrigami(Origami((1,4)(2,6,3), (1,5)(2,3,6,4), 6), Origami((1,4,3)
          >(2,5), (1,5,3,4)(2,6), 6));
          true
  
          gap> EquivalentOrigami(Origami((1,4)(2,6,3), (1,5)(2,3,6,4), 6), Origami((1,2,5)
          >(3,4,6), (1,2)(5,6), 6));
          false
          
\end{Verbatim}
 }

 }

 }

 \def\indexname{Index\logpage{[ "Ind", 0, 0 ]}
\hyperdef{L}{X83A0356F839C696F}{}
}

\cleardoublepage
\phantomsection
\addcontentsline{toc}{chapter}{Index}


\printindex

\newpage
\immediate\write\pagenrlog{["End"], \arabic{page}];}
\immediate\closeout\pagenrlog
\end{document}
