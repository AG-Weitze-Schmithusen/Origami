% generated by GAPDoc2LaTeX from XML source (Frank Luebeck)
\documentclass[a4paper,11pt]{report}

\usepackage{a4wide}
\sloppy
\pagestyle{myheadings}
\usepackage{amssymb}
\usepackage[latin1]{inputenc}
\usepackage{makeidx}
\makeindex
\usepackage{color}
\definecolor{FireBrick}{rgb}{0.5812,0.0074,0.0083}
\definecolor{RoyalBlue}{rgb}{0.0236,0.0894,0.6179}
\definecolor{RoyalGreen}{rgb}{0.0236,0.6179,0.0894}
\definecolor{RoyalRed}{rgb}{0.6179,0.0236,0.0894}
\definecolor{LightBlue}{rgb}{0.8544,0.9511,1.0000}
\definecolor{Black}{rgb}{0.0,0.0,0.0}

\definecolor{linkColor}{rgb}{0.0,0.0,0.554}
\definecolor{citeColor}{rgb}{0.0,0.0,0.554}
\definecolor{fileColor}{rgb}{0.0,0.0,0.554}
\definecolor{urlColor}{rgb}{0.0,0.0,0.554}
\definecolor{promptColor}{rgb}{0.0,0.0,0.589}
\definecolor{brkpromptColor}{rgb}{0.589,0.0,0.0}
\definecolor{gapinputColor}{rgb}{0.589,0.0,0.0}
\definecolor{gapoutputColor}{rgb}{0.0,0.0,0.0}

%%  for a long time these were red and blue by default,
%%  now black, but keep variables to overwrite
\definecolor{FuncColor}{rgb}{0.0,0.0,0.0}
%% strange name because of pdflatex bug:
\definecolor{Chapter }{rgb}{0.0,0.0,0.0}
\definecolor{DarkOlive}{rgb}{0.1047,0.2412,0.0064}


\usepackage{fancyvrb}

\usepackage{mathptmx,helvet}
\usepackage[T1]{fontenc}
\usepackage{textcomp}


\usepackage[
            pdftex=true,
            bookmarks=true,        
            a4paper=true,
            pdftitle={Written with GAPDoc},
            pdfcreator={LaTeX with hyperref package / GAPDoc},
            colorlinks=true,
            backref=page,
            breaklinks=true,
            linkcolor=linkColor,
            citecolor=citeColor,
            filecolor=fileColor,
            urlcolor=urlColor,
            pdfpagemode={UseNone}, 
           ]{hyperref}

\newcommand{\maintitlesize}{\fontsize{50}{55}\selectfont}

% write page numbers to a .pnr log file for online help
\newwrite\pagenrlog
\immediate\openout\pagenrlog =\jobname.pnr
\immediate\write\pagenrlog{PAGENRS := [}
\newcommand{\logpage}[1]{\protect\write\pagenrlog{#1, \thepage,}}
%% were never documented, give conflicts with some additional packages

\newcommand{\GAP}{\textsf{GAP}}

%% nicer description environments, allows long labels
\usepackage{enumitem}
\setdescription{style=nextline}

%% depth of toc
\setcounter{tocdepth}{1}





%% command for ColorPrompt style examples
\newcommand{\gapprompt}[1]{\color{promptColor}{\bfseries #1}}
\newcommand{\gapbrkprompt}[1]{\color{brkpromptColor}{\bfseries #1}}
\newcommand{\gapinput}[1]{\color{gapinputColor}{#1}}


\begin{document}

\logpage{[ 0, 0, 0 ]}
\begin{titlepage}
\mbox{}\vfill

\begin{center}{\maintitlesize \textbf{The \textsf{ThreeKPlusOne} Package\mbox{}}}\\
\vfill

\hypersetup{pdftitle=The \textsf{ThreeKPlusOne} Package}
\markright{\scriptsize \mbox{}\hfill The \textsf{ThreeKPlusOne} Package \hfill\mbox{}}
{\Huge Version 1\mbox{}}\\[1cm]
\mbox{}\\[2cm]
{\Large \textbf{Pascal Kattler  \mbox{}}}\\
\hypersetup{pdfauthor=Pascal Kattler  }
\end{center}\vfill

\mbox{}\\
{\mbox{}\\
\small \noindent \textbf{Pascal Kattler  }  Email: \href{mailto://kattler@math.uni-sb.de} {\texttt{kattler@math.uni-sb.de}}}\\
\end{titlepage}

\newpage\setcounter{page}{2}
\newpage

\def\contentsname{Contents\logpage{[ 0, 0, 1 ]}}

\tableofcontents
\newpage

 
\chapter{\textcolor{Chapter }{The $3k+1$ Problem}}\logpage{[ 1, 0, 0 ]}
\hyperdef{L}{X8769FB038733F6F4}{}
{
  
\section{\textcolor{Chapter }{Introduction}}\label{sec:theory}
\logpage{[ 1, 1, 0 ]}
\hyperdef{L}{X7DFB63A97E67C0A1}{}
{
  This package provides calculations with Origamis. An Origami can be obtained
in the following way from two permutations $\sigma_a, \sigma_b \in S_d$. We take $d$ Squares $Q_1, \dots, Q_d$ and clue the lower side of $Q_i$ to the upper side of $Q_{\sigma_y(i)}$ and the right side of $Q_i$ to the left side of $Q_{\sigma_x(i)}$. So in this Package we identify an Origami with a pair of permutations, witch
acts transitive on $\{1 \dots d\}$ up to simultan conjugation. We store an Origami as Origami object. We are
mainly interested in the Veechgroup of an Origami. It can be shown that the
Veechgroup of an Origami is a subgroup of $Sl_2(\mathbb{Z})$ of finite index. So we fix two generators 
\[ S = \left( {\begin{array}{cc} 0 & -1 \\ 1 & 0 \\ \end{array} } \right) \]
 and 
\[ T = \left( {\begin{array}{cc} 1 & 1 \\ 0 & 1 \\ \end{array} } \right). \]
 }

 
\section{\textcolor{Chapter }{ The Free Group }}\logpage{[ 1, 2, 0 ]}
\hyperdef{L}{X7EFD570179465DD5}{}
{
  In this package we fix the Free Group $F$ generated by $S$ and $T$. }

 
\section{\textcolor{Chapter }{The Origmai Object}}\logpage{[ 1, 3, 0 ]}
\hyperdef{L}{X7DCBDD6F8173A517}{}
{
  In this section we describe the main function of this package. 

\subsection{\textcolor{Chapter }{ActionOfSl}}
\logpage{[ 1, 3, 1 ]}\nobreak
\hyperdef{L}{X7C8464AA834B03C3}{}
{\noindent\textcolor{FuncColor}{$\triangleright$\enspace\texttt{ActionOfSl({\mdseries\slshape word, Origami})\index{ActionOfSl@\texttt{ActionOfSl}}
\label{ActionOfSl}
}\hfill{\scriptsize (function)}}\\


 This function lets act a word in the free group $Group(S, T)$ ,representing an element of $Sl_2 (\mathbb{Z})$ on an Origami and returns $word.Origami$. 
\begin{Verbatim}[commandchars=!@|,fontsize=\small,frame=single,label=Example]
  !gapprompt@gap>| !gapinput@ActionOfSl(S*T,Origami((1,3,5), (1,3)(2,4,5), 5));|
  Origami((1,3)(2,5,4), (2,4,5,3), 5)
\end{Verbatim}
 \noindent\textcolor{FuncColor}{$\triangleright$\enspace\texttt{ActionOfF2ViaCanonical({\mdseries\slshape Origami, word})\index{ActionOfF2ViaCanonical@\texttt{ActionOfF2ViaCanonical}}
\label{ActionOfF2ViaCanonical}
}\hfill{\scriptsize (function)}}\\


 This lets act a word in the free group $Group(S, T)$, representing an element of $Sl_2(\mathbb{Z})$, on an Origami and returns $word.Origami$. But in contrast to \texttt{ActionOfSl} the result is stored in the canonical representation. ATTENTION: the order of
arguments is here reserved. 
\begin{Verbatim}[commandchars=!@|,fontsize=\small,frame=single,label=Example]
    gap> ActionOfF2ViaCanonical(Origami((1,2), (1,3), 3), S);
  Origami((1,2), (2,3), 3)
    
\end{Verbatim}
 \noindent\textcolor{FuncColor}{$\triangleright$\enspace\texttt{RightActionOfF2ViaCanonical({\mdseries\slshape Origami, word})\index{RightActionOfF2ViaCanonical@\texttt{RightActionOfF2ViaCanonical}}
\label{RightActionOfF2ViaCanonical}
}\hfill{\scriptsize (function)}}\\


 This lets act a word in the free group $Group(S, T)$ on an Origami from right and returns $Origami.word = word^-1.Origami$, where the left action is the common action of $Sl_2(\mathbb{Z})$ on 2 mannifolds. This action has the same Veechgroup and orbits as the left
action. In contrast to \texttt{ActionOfSl} the result is stored in the canonical representation. ATTENTION: the order of
arguments is here reserved. 
\begin{Verbatim}[commandchars=!@|,fontsize=\small,frame=single,label=Example]
  !gapprompt@gap>| !gapinput@RightActionOfF2ViaCanonical(Origami((2,3), (1,3,2), 3),T);|
  Origami((1,2), (2,3), 3)
   
\end{Verbatim}
 \noindent\textcolor{FuncColor}{$\triangleright$\enspace\texttt{CanonicalOrigamiViaDelecroixAndStart({\mdseries\slshape Origami, start})\index{CanonicalOrigamiViaDelecroixAndStart@\texttt{Canonical}\-\texttt{Origami}\-\texttt{Via}\-\texttt{Delecroix}\-\texttt{And}\-\texttt{Start}}
\label{CanonicalOrigamiViaDelecroixAndStart}
}\hfill{\scriptsize (function)}}\\


 This calculates a canonical representation of an origami depending on a given
number start (Between 1 and the degree of of the Origami). To determine a
canonical numbering the algorithm starts at the square with number start and
walks over the origami in a certain way and numbers the squares in the order,
they are visited . First it walks in horizontal direction one loop. Then it
walks one step up (in vertical direection) and then again a loop in horizontal
direction. This wil be repeated until the vertical loop is complete or all
squares have been visited. If there are unvisited squares, we continue with
the smallest number (in the new numbering), that has not been in a vertical
loop. An Origami is connected, so that number exists. Two origamis are equal
if they are described by the same permutations in their canonical
representation. 
\begin{Verbatim}[commandchars=!@|,fontsize=\small,frame=single,label=Example]
  !gapprompt@gap>| !gapinput@CanonicalOrigamiViaDelecroixAndStart(Origami((1,10,7,6,8,9,2)(4,5),|
  !gapprompt@gap>| !gapinput@(1,8,5,4)(2,10,6)(3,9,7), 10), 1);|
  Origami((1,2,3,4,5,6,7)(8,9), (1,5,8,9)(2,4,7)(3,10,6), 10)
\end{Verbatim}
 \noindent\textcolor{FuncColor}{$\triangleright$\enspace\texttt{CanonicalOrigamiViaDelecroix({\mdseries\slshape Origami})\index{CanonicalOrigamiViaDelecroix@\texttt{CanonicalOrigamiViaDelecroix}}
\label{CanonicalOrigamiViaDelecroix}
}\hfill{\scriptsize (function)}}\\


 This calculates a canonical representation of an origami. It calculates the
representation from CanonicalOrigamiViaDelecroixAndStart with several start
squares, independent of the given representation. Then it takes the minimum
with respect to some order. Two origamis are equal if they are described by
the same permutations in their canonical representation. 
\begin{Verbatim}[commandchars=!@|,fontsize=\small,frame=single,label=Example]
  !gapprompt@gap>| !gapinput@CanonicalOrigamiViaDelecroix(Origami((1,10,7,6,8,9,2)(4,5), (1,8,5,4)(2,10,6)(3,9,7), 10));|
  Origami((2,3,4,5,6,7,8)(9,10), (1,2,6)(3,5,7)(4,8,9,10), 10)
\end{Verbatim}
 \noindent\textcolor{FuncColor}{$\triangleright$\enspace\texttt{CanonicalOrigami({\mdseries\slshape Origami})\index{CanonicalOrigami@\texttt{CanonicalOrigami}}
\label{CanonicalOrigami}
}\hfill{\scriptsize (function)}}\\


 This calculates a canonical representation of an origami. Two origamis are
equal if they are described by the same permutations in their canonical
representation. This is an older slower Version. 
\begin{Verbatim}[commandchars=!@|,fontsize=\small,frame=single,label=Example]
    gap> CanonicalOrigami(Origami((1,10,7,6,8,9,2)(4,5), (1,8,5,4)(2,10,6)(3,9,7), 10));
  Origami((1,2)(3,4,5,6,7,8,9), (1,2,3,7)(4,6,9)(5,10,8), 10)
\end{Verbatim}
 \noindent\textcolor{FuncColor}{$\triangleright$\enspace\texttt{OrigamiFamily\index{OrigamiFamily@\texttt{OrigamiFamily}}
\label{OrigamiFamily}
}\hfill{\scriptsize (family)}}\\


 The only sense of this Familiy is, that Origami does not fit in any other \noindent\textcolor{FuncColor}{$\triangleright$\enspace\texttt{HorizontalPerm({\mdseries\slshape Origami})\index{HorizontalPerm@\texttt{HorizontalPerm}}
\label{HorizontalPerm}
}\hfill{\scriptsize (attribute)}}\\


 This returns the vertical permutation $\sigma_y$ of the Origami. 
\begin{Verbatim}[commandchars=!@|,fontsize=\small,frame=single,label=Example]
            gap> HorizontalPerm(Origami((1,3,5), (1,3)(2,4,5), 5));
  (1,3,5)
          
\end{Verbatim}
 \noindent\textcolor{FuncColor}{$\triangleright$\enspace\texttt{VerticalPerm({\mdseries\slshape Origami})\index{VerticalPerm@\texttt{VerticalPerm}}
\label{VerticalPerm}
}\hfill{\scriptsize (attribute)}}\\


 This returns the horizontal permutation $\sigma_x$ of the Origami. 
\begin{Verbatim}[commandchars=!@|,fontsize=\small,frame=single,label=Example]
            gap> VerticalPerm( Origami((1,3,5), (1,3)(2,4,5), 5));
              (1,3)(2,4,5)
          
\end{Verbatim}
 \noindent\textcolor{FuncColor}{$\triangleright$\enspace\texttt{DegreeOrigami({\mdseries\slshape Origami})\index{DegreeOrigami@\texttt{DegreeOrigami}}
\label{DegreeOrigami}
}\hfill{\scriptsize (attribute)}}\\


 This returns the degree of an Origami. 
\begin{Verbatim}[commandchars=!@|,fontsize=\small,frame=single,label=Example]
            gap> DegreeOrigami(Origami((1,3,5), (1,3)(2,4,5), 5));
  5
          
\end{Verbatim}
 \noindent\textcolor{FuncColor}{$\triangleright$\enspace\texttt{Stratum({\mdseries\slshape Origami})\index{Stratum@\texttt{Stratum}}
\label{Stratum}
}\hfill{\scriptsize (attribute)}}\\


 This calculates the stratum of an Origami. That is a list of the orders of the
singularities 
\begin{Verbatim}[commandchars=!@|,fontsize=\small,frame=single,label=Example]
            gap> Stratum(Origami((1,6,4,7,5,3)(2,8), (1,4,5,3,8,2,6), 8));
  [ 1, 5 ]
          
\end{Verbatim}
 \noindent\textcolor{FuncColor}{$\triangleright$\enspace\texttt{Genus({\mdseries\slshape Origami})\index{Genus@\texttt{Genus}}
\label{Genus}
}\hfill{\scriptsize (attribute)}}\\


 This calculates the genus of the Origami surface. 
\begin{Verbatim}[commandchars=!@|,fontsize=\small,frame=single,label=Example]
            gap> Genus( Origami((1,2,3,4),(1,2)(3,4), 4) );
  2
          
\end{Verbatim}
 \noindent\textcolor{FuncColor}{$\triangleright$\enspace\texttt{VeechGroup({\mdseries\slshape Origami})\index{VeechGroup@\texttt{VeechGroup}}
\label{VeechGroup}
}\hfill{\scriptsize (attribute)}}\\


 This calculates the Veechgroup of an Origami. This is a subgroup of $Sl_2(\mathbb{Z})$ of finite degree. The group is stored as ModularSubgroup from the \textsf{ModularSubgroup} package. The Veechgroup is represented as the coset permutations $\sigma_S$ and $\sigma_T$ with respect to the generators $S$ and $T$. This means if $i$ is the integer associated to the rigth coset $G$ (Cosets( O ) [ i ] VeechGroup = H) then we have for the coset $H$, associated to $\sigma_S(i)$, that $SG = H$. Dito for $\sigma_T$. You get the coset Permutations from the ModularSubgroup like in the
following Example. 
\begin{Verbatim}[commandchars=!@|,fontsize=\small,frame=single,label=Example]
  !gapprompt@gap>| !gapinput@SAction(VeechGroup(Origami((1,2,5)(3,4,6), (1,2)(5,6), 6)));|
  (1,3)(2,5)(4,7)(6,8)(9,10)
  !gapprompt@gap>| !gapinput@TAction(VeechGroup(Origami((1,2,5)(3,4,6), (1,2)(5,6), 6)));|
  (1,2,4)(3,6)(5,8,7,9,10)
          
\end{Verbatim}
 \noindent\textcolor{FuncColor}{$\triangleright$\enspace\texttt{Cosets({\mdseries\slshape Origami})\index{Cosets@\texttt{Cosets}}
\label{Cosets}
}\hfill{\scriptsize (attribute)}}\\


 This Calculates the right cosets of the Veechgroup of an Origami as list of
words in $S$ and $T$. 
\begin{Verbatim}[commandchars=!@|,fontsize=\small,frame=single,label=Example]
            gap> Cosets(Origami((1,2,5)(3,4,6), (1,2)(5,6), 6));
  [ < identity ...>, T, S, T^2, T*S, S*T, T^2*S, T*S*T, T^2*S*T, T^2*S*T^2 ]
          
\end{Verbatim}
 \noindent\textcolor{FuncColor}{$\triangleright$\enspace\texttt{({\mdseries\slshape Origami1, Origami2})\index{@\texttt{}}
\label{}
}\hfill{\scriptsize (function)}}\\


 This tests wether Origami1 is equal up to Origami2 up to numbering of the
squares. 
\begin{Verbatim}[commandchars=!@|,fontsize=\small,frame=single,label=Example]
  !gapprompt@gap>| !gapinput@EquivalentOrigami(Origami((1,4)(2,6,3), (1,5)(2,3,6,4), 6), Origami((1,4,3)(2,5), (1,5,3,4)(2,6), 6));|
  true
  
  !gapprompt@gap>| !gapinput@EquivalentOrigami(Origami((1,4)(2,6,3), (1,5)(2,3,6,4), 6), Origami((1,2,5)(3,4,6), (1,2)(5,6), 6));|
  false
          
\end{Verbatim}
 }

 }

 }

 \def\indexname{Index\logpage{[ "Ind", 0, 0 ]}
\hyperdef{L}{X83A0356F839C696F}{}
}

\cleardoublepage
\phantomsection
\addcontentsline{toc}{chapter}{Index}


\printindex

\newpage
\immediate\write\pagenrlog{["End"], \arabic{page}];}
\immediate\closeout\pagenrlog
\end{document}
